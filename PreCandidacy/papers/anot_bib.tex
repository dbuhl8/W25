\documentclass{article}

\usepackage{graphicx} % Required for inserting images
\usepackage[left=1in,right=1in,top=1in,bottom=1in]{geometry}
\usepackage{amsmath}
\usepackage{amsthm} %proof environment
\usepackage{amssymb}
\usepackage{amsfonts}
\usepackage{enumitem} %nice lists
\usepackage{verbatim} %useful for something 
\usepackage{xcolor}
\usepackage{setspace}
\usepackage{blindtext} % I have no idea what this is 
\usepackage{caption}  % need this for unnumbered captions/figures
\usepackage{natbib}
\usepackage{tikz}
\usepackage{hyperref}

\begin{document}

\title{Pre-candidacy notes: }
\author{Dante Buhl}

\newcommand{\wrms}{w_{\text{rms}}}
\newcommand{\bs}[1]{\boldsymbol{#1}}
\newcommand{\tb}[1]{\textbf{#1}}
\newcommand{\bmp}[1]{\begin{minipage}{#1\textwidth}}
\newcommand{\emp}{\end{minipage}}
\newcommand{\R}{\mathbb{R}}
\newcommand{\C}{\mathbb{C}}
\newcommand{\N}{\mathcal{N}}
\newcommand{\K}{\bs{\mathrm{K}}}
\newcommand{\m}{\bs{\mu}_*}
\newcommand{\s}{\bs{\Sigma}_*}
\newcommand{\dt}{\Delta t}
\newcommand{\dx}{\Delta x}
\newcommand{\tr}[1]{\text{Tr}(#1)}
\newcommand{\Tr}[1]{\text{Tr}(#1)}
\newcommand{\Div}{\nabla \cdot}
\renewcommand{\div}{\nabla \cdot}
\newcommand{\Curl}{\nabla \times}
\newcommand{\Grad}{\nabla}
\newcommand{\grad}{\nabla}
\newcommand{\grads}{\nabla_s}
\newcommand{\gradf}{\nabla_f}
\newcommand{\xs}{x_s}
\newcommand{\xf}{x_f}
\newcommand{\ts}{t_s}
\newcommand{\tf}{t_f}
\newcommand{\pt}{\partial t}
\newcommand{\pz}{\partial z}
\newcommand{\uvec}{\bs{u}}
\newcommand{\F}{\bs{F}}
\newcommand{\T}{\tilde{T}}
\newcommand{\ez}{\bs{e}_z}
\newcommand{\ex}{\bs{e}_x}
\newcommand{\ey}{\bs{e}_y}
\newcommand{\eo}{\bs{e}_{\bs{\Omega}}}
\newcommand{\ppt}[1]{\frac{\partial #1}{\partial t}}
\newcommand{\DDt}[1]{\frac{D #1}{D t}}
\newcommand{\ppts}[1]{\frac{\partial #1}{\partial t_s}}
\newcommand{\pptf}[1]{\frac{\partial #1}{\partial t_f}}
\newcommand{\ppz}[1]{\frac{\partial #1}{\partial z}}
\newcommand{\ddz}[1]{\frac{d #1}{d z}}
\newcommand{\ppzetas}[1]{\frac{\partial^2 #1}{\partial \zeta^2}}
\newcommand{\ppzs}[1]{\frac{\partial #1}{\partial z_s}}
\newcommand{\ppzf}[1]{\frac{\partial #1}{\partial z_f}}
\newcommand{\ppx}[1]{\frac{\partial #1}{\partial x}}
\newcommand{\ppxi}[1]{\frac{\partial #1}{\partial x_i}}
\newcommand{\ppxj}[1]{\frac{\partial #1}{\partial x_j}}
\newcommand{\ppy}[1]{\frac{\partial #1}{\partial y}}
\newcommand{\ppzeta}[1]{\frac{\partial #1}{\partial \zeta}}


\maketitle 
% This line removes the automatic indentation on new paragraphs
\setlength{\parindent}{0pt}

\section{Billant \& Chomaz Papers}

\subsection{Self-similarity of strongly stratified inviscid flows (2001)}

\begin{itemize}
    \item Posits the scaling of an intrinsic vertical length scale of strongly
    stratified flows, $l_z \propto U/N$. 
    \item Third paper which describes the ``Zig-Zag'' instability. Two previous
    papers conducted linear stability analysis of the instability. 
    \item Zig-zag instability is self-similar with respect to $k_zU/N$ which
    implies that the dominant vertical wavenumber of the flow is proportional to
    $Fr$. 
\end{itemize}


\section{Hattori \& Hirota Papers}

\subsection{Stability of two-dimensional Taylor-Green vortices in rotating
stratified fluids (2023)}
\begin{itemize}
    \item Conducted a local stability analysis as well as DNS and analyzed the
    data using modal stability analysis. 
    \item Linear Stability analysis is conducted on a linearized and inviscid
    version of the governing equations.
    \item Both the DNS and LSA begin with a base flow composed of Taylor-Green
    vortices, which are arranged in a grid lattice. 
    \item 5 instabilities are identified from the LSA, each with a different
    mechanism and different instability/resonance conditions. 
    \item 
\end{itemize}

\subsection{Modal stability analysis of arrays of stably stratified vortices
(2021)}
\begin{itemize}
    \item 
\end{itemize}

\section{GFD Group (Garaud, Chini, Shah, Caulfield \ldots)}

\subsection{Exploiting self-organized criticality in strongly stratified
turbulence (2021)}
\begin{itemize}
    \item Developed a multiscale model for strongly stratified flows wherein an
    aspect ratio $\alpha$ is used to describe scale separation of horizontal and
    vertical motions recovering that $l_z \propto Fr$ as posited by
    \cite{BillantChomaz2001}. 
\end{itemize}

\subsection{Cope et al. 2020}

\subsection{Shah et al. 2023}

\colorbox{yellow}{Add a bibliography using natbib. Need to build a bib file for this}. 

\end{document}
