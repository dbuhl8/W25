\documentclass{article}

\usepackage{graphicx} % Required for inserting images
\usepackage[left=1in,right=1in,top=1in,bottom=1in]{geometry}
\usepackage{amsmath}
\usepackage{amsthm} %proof environment
\usepackage{amssymb}
\usepackage{amsfonts}
\usepackage{enumitem} %nice lists
\usepackage{verbatim} %useful for something 
\usepackage{xcolor}
\usepackage{setspace}
\usepackage{blindtext} % I have no idea what this is 
\usepackage{caption}  % need this for unnumbered captions/figures
\usepackage{natbib}
\usepackage{tikz}
\usepackage{hyperref}

\begin{document}

\title{Pre-candidacy notes: }
\author{Dante Buhl}

\newcommand{\wrms}{w_{\text{rms}}}
\newcommand{\bs}[1]{\boldsymbol{#1}}
\newcommand{\tb}[1]{\textbf{#1}}
\newcommand{\bmp}[1]{\begin{minipage}{#1\textwidth}}
\newcommand{\emp}{\end{minipage}}
\newcommand{\R}{\mathbb{R}}
\newcommand{\C}{\mathbb{C}}
\newcommand{\N}{\mathcal{N}}
\newcommand{\K}{\bs{\mathrm{K}}}
\newcommand{\m}{\bs{\mu}_*}
\newcommand{\s}{\bs{\Sigma}_*}
\newcommand{\dt}{\Delta t}
\newcommand{\dx}{\Delta x}
\newcommand{\tr}[1]{\text{Tr}(#1)}
\newcommand{\Tr}[1]{\text{Tr}(#1)}
\newcommand{\Div}{\nabla \cdot}
\renewcommand{\div}{\nabla \cdot}
\newcommand{\Curl}{\nabla \times}
\newcommand{\Grad}{\nabla}
\newcommand{\grad}{\nabla}
\newcommand{\grads}{\nabla_s}
\newcommand{\gradf}{\nabla_f}
\newcommand{\xs}{x_s}
\newcommand{\xf}{x_f}
\newcommand{\ts}{t_s}
\newcommand{\tf}{t_f}
\newcommand{\pt}{\partial t}
\newcommand{\pz}{\partial z}
\newcommand{\uvec}{\bs{u}}
\newcommand{\F}{\bs{F}}
\newcommand{\T}{\tilde{T}}
\newcommand{\ez}{\bs{e}_z}
\newcommand{\ex}{\bs{e}_x}
\newcommand{\ey}{\bs{e}_y}
\newcommand{\eo}{\bs{e}_{\bs{\Omega}}}
\newcommand{\ppt}[1]{\frac{\partial #1}{\partial t}}
\newcommand{\DDt}[1]{\frac{D #1}{D t}}
\newcommand{\ppts}[1]{\frac{\partial #1}{\partial t_s}}
\newcommand{\pptf}[1]{\frac{\partial #1}{\partial t_f}}
\newcommand{\ppz}[1]{\frac{\partial #1}{\partial z}}
\newcommand{\ddz}[1]{\frac{d #1}{d z}}
\newcommand{\ppzetas}[1]{\frac{\partial^2 #1}{\partial \zeta^2}}
\newcommand{\ppzs}[1]{\frac{\partial #1}{\partial z_s}}
\newcommand{\ppzf}[1]{\frac{\partial #1}{\partial z_f}}
\newcommand{\ppx}[1]{\frac{\partial #1}{\partial x}}
\newcommand{\ppxi}[1]{\frac{\partial #1}{\partial x_i}}
\newcommand{\ppxj}[1]{\frac{\partial #1}{\partial x_j}}
\newcommand{\ppy}[1]{\frac{\partial #1}{\partial y}}
\newcommand{\ppzeta}[1]{\frac{\partial #1}{\partial \zeta}}


\maketitle 
% This line removes the automatic indentation on new paragraphs
\setlength{\parindent}{0pt}

\section{Billant \& Chomaz Papers}

\subsection{Experimental evidence for a new instability of a
vertical columnar vortex pair in a strongly
stratified fluid (2000)}
\begin{itemize}
    \item The first paper in a series of papers by Billant and Chomaz describing
    and investigating the properties of the so called ``zigzag'' instability
    present in the Lamb-Chaplygin vortex pair (a counterrotating vortex dipole). 
    \item This paper demonstrated the existence of such an instability from
    experimental findings at sufficient stratification. For insufficient
    stratification $Fr \ge 0.2 \pm 0.01$, the ellipitcal instability appears to
    be the dominant instability and after its gravitational collapse, the vortex
    pair appears to irregularly zigzag into layer formation. 
    \item From what can be observed from the zigzag instability is that it
    doesn't perturb the horizontal cross-section structure of the vortex column,
    only its vertical structure. It is positted that this phenomenon may be
    responsible for the layering phenomenon demonstrated in many stratified
    flows. 
    \item Over a long enough time frame the original vortex pair column ends up
    divided into pancake dipole segments in the vertical direction, obtaining
    what is usually described as pancake eddies in the flow. 
\end{itemize}

\subsection{Self-similarity of strongly stratified inviscid flows (2001)}

\begin{itemize}
    \item Posits the scaling of an intrinsic vertical length scale of strongly
    stratified flows, $l_z \propto U/N$. 
    \item Third paper which describes the ``Zig-Zag'' instability. Two previous
    papers conducted linear stability analysis of the instability. 
    \item Zig-zag instability is self-similar with respect to $k_zU/N$ which
    implies that the dominant vertical wavenumber of the flow is proportional to
    $Fr$. 
\end{itemize}

\subsection{Three-dimensional stability of vertical columnar vortex pair in a
stratified fluid}

\begin{itemize}
    \item This paper conducted a numerical stability analysis on the linearized
    equations using mean-pertubation flow separation. They found for flows
    with sufficient stratification that the primary instability of the
    counterrotating vortex pair was the ``zig-zag'' instability in which the
    entire vortex column was destabilized and oscillationed ide to side with a
    typical scale height, later found to be proportional to the froude number.
    For insufficient stratification, the elliptical instability was the dominant
    instabtility. 
    \item Among their findings is the approximate scaling that the root mean
    squared $u_z' \propto
    1/Fr$ and $p' \propto Fr$ (normalized by the rms horizontal velocity).
    Furthermore, these numerical findings for the growth rate of the zig-zag
    instability concur with the experimental results
    within reasonable error. 
    \item Their nondimensionalization involved 
    \begin{gather*}
        Fr = \frac{U_{\text{prop}}}{NR}
    \end{gather*}
    where $U_{\text{prop}}$ is the propagation speed of the vortex pair, and $R$
    is given by the dipole radius. This is similar to the non-dimensionalization
    from Chini et al, in which the unit velocity and lengthscale are given by
    the typical horizontal flow (i.e. horizontal forcing which is order 1 in
    both $U$ and $L$). 
\end{itemize}


\section{Hattori \& Hirota Papers}

\subsection{Stability of two-dimensional Taylor-Green vortices in rotating
stratified fluids (2023)}
\begin{itemize}
    \item Conducted a local stability analysis as well as DNS and analyzed the
    data using modal stability analysis. 
    \item Linear Stability analysis is conducted on a linearized and inviscid
    version of the governing equations.
    \item Both the DNS and LSA begin with a base flow composed of Taylor-Green
    vortices, which are arranged in a grid lattice. 
    \item 5 instabilities are identified from the LSA, each with a different
    mechanism and different instability/resonance conditions. 
    \item Linear Stability analysis found that the pure hyperbolic instability
    is often the fastest growing istability as also the most realizable.
    Variation of the input rossby and froude numbers reveals characteristics of
    other secondary instabilities which vary with vertical wavenumber, and
    radius from vortex centers (as well as input parameters). 
\end{itemize}

\subsection{Modal stability analysis of arrays of stably stratified vortices
(2021)}
\begin{itemize}
    \item 
\end{itemize}


\section{Miyazaki and Fukumoto}

\subsection{Three‐dimensional instability of strained vortices in a stably
stratified fluid (1992)}
\begin{itemize}
    \item This paper conducts a linear stability analysis of ``unbounded
    strained vortices''. The linear stability analysis is derived analytically
    and then solved numerically by a floquet problem. The primary investigation
    is into the elliptical instability of ``Pierrehumbertt type''. Two other
    instability modes are noted which depend distinctly on the buoyancy
    frequency $N$.
\end{itemize}

\subsection{Elliptical instability in a stably stratified rotating fluid}
\begin{itemize}
   \item stuff 
\end{itemize}


\section{Herring and Metias}

\subsection{}
\begin{itemize}
    \item stuff
\end{itemize}

\section{Waite and Bartello}
\subsection{}
\begin{itemize}
    \item stuff
\end{itemize}


\section{GFD Group (Garaud, Chini, Shah, Caulfield \ldots)}

\subsection{Exploiting self-organized criticality in strongly stratified
turbulence (2021)}
\begin{itemize}
    \item Developed a multiscale model for strongly stratified flows wherein an
    aspect ratio $\alpha$ is used to describe scale separation of horizontal and
    vertical motions recovering that $l_z \propto Fr$ as posited by
    %\cite{BillantChomaz2001}. 
\end{itemize}

\subsection{Cope et al. 2020}

\subsection{Shah et al. 2023}

\colorbox{yellow}{Add a bibliography using natbib. Need to build a bib file for this}. 

\end{document}
