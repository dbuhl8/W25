
\documentclass[lineno]{jfm}

\usepackage{graphicx}
\usepackage{newtxtext}
\usepackage{newtxmath}
%\usepackage{amsmath}
%\usepackage{amsthm} %proof environment
%\usepackage{amssymb}
%\usepackage{amsfonts}
%\usepackage{enumitem} %nice lists
%\usepackage{verbatim} %useful for something 
\usepackage{xcolor}
%\usepackage{setspace}
%\usepackage{blindtext} % I have no idea what this is 
%\usepackage{caption}  % need this for unnumbered captions/figures
%\usepackage{tikz}
%\usepackage{soul} % need this for the hl command
\usepackage{natbib}
\usepackage{hyperref}
\hypersetup{
    colorlinks = true,
    urlcolor   = blue,
    citecolor  = black,
}
\newtheorem{lemma}{Lemma}
\newtheorem{corollary}{Corollary}
\newcommand{\RomanNumeralCaps}[1]
\linenumbers

\title{Precandidacy Exam: Report}

\author{Dante Buhl\aff{1}
  \corresp{\email{dbuhl@ucsc.edu}}}

\affiliation{\aff{1} Department of Applied Mathematics, Baskin School of
Engineering, Univeristy of California
Santa Cruz, 1156 High Street, Santa Cruz, CA, 95064 US}

\begin{document}

\newcommand{\red}{\color{red}}
\newcommand{\wrms}{w_{\text{rms}}}
\newcommand{\bs}[1]{\boldsymbol{#1}}
\newcommand{\tb}[1]{\textbf{#1}}
\newcommand{\bmp}[1]{\begin{minipage}{#1\textwidth}}
\newcommand{\emp}{\end{minipage}}
\newcommand{\R}{\mathbb{R}}
\newcommand{\C}{\mathbb{C}}
\newcommand{\N}{\mathcal{N}}
\newcommand{\K}{\bs{\mathrm{K}}}
\newcommand{\m}{\bs{\mu}_*}
\newcommand{\s}{\bs{\Sigma}_*}
\newcommand{\dt}{\Delta t}
\newcommand{\dx}{\Delta x}
\newcommand{\tr}[1]{\text{Tr}(#1)}
\newcommand{\Tr}[1]{\text{Tr}(#1)}
\newcommand{\Div}{\nabla \cdot}
\renewcommand{\div}{\nabla \cdot}
\newcommand{\Curl}{\nabla \times}
\newcommand{\Grad}{\nabla}
\newcommand{\grad}{\nabla}
\newcommand{\grads}{\nabla_s}
\newcommand{\gradf}{\nabla_f}
\newcommand{\xs}{\bs{x}_s}
\newcommand{\xf}{\bs{x}_f}
\newcommand{\ts}{t_s}
\newcommand{\tf}{t_f}
\newcommand{\pt}{\partial t}
\newcommand{\pz}{\partial z}
\newcommand{\uvec}{\bs{u}}
\newcommand{\F}{\bs{F}}
\newcommand{\T}{\tilde{T}}
\newcommand{\ez}{\bs{e}_z}
\newcommand{\ex}{\bs{e}_x}
\newcommand{\ey}{\bs{e}_y}
\newcommand{\eo}{\bs{e}_{\bs{\Omega}}}
\newcommand{\ppt}[1]{\frac{\partial #1}{\partial t}}
\newcommand{\ppts}[1]{\frac{\partial #1}{\partial t_s}}
\newcommand{\pptf}[1]{\frac{\partial #1}{\partial t_f}}
\newcommand{\ppz}[1]{\frac{\partial #1}{\partial z}}
\newcommand{\ddz}[1]{\frac{d #1}{d z}}
\newcommand{\ppzetas}[1]{\frac{\partial^2 #1}{\partial \zeta^2}}
\newcommand{\ppzs}[1]{\frac{\partial #1}{\partial z_s}}
\newcommand{\ppzf}[1]{\frac{\partial #1}{\partial z_f}}
\newcommand{\ppx}[1]{\frac{\partial #1}{\partial x}}
\newcommand{\ppy}[1]{\frac{\partial #1}{\partial y}}
\newcommand{\ppzeta}[1]{\frac{\partial #1}{\partial \zeta}}

\maketitle

%\begin{abstract}
    %stuff here
%\end{abstract}

%\begin{keywords}
    %stuff here
%\end{keywords}

\section{Introduction}
\label{sec:intro}

In the recent history of fluid dynamics, few studies have completely acclimated
the dynamics of both stratification and rotation in a fully comprehensive
manner. By contrast, many achievements have been made in studying isolated
dynamics, whereby the effects of other physical mechanisms are ignored in order to
better understand the instabilities, flow structures, and other properties which arise
under specific conditions. This has led to many discoveries about stratified flows ({\red CITE PAPERS HERE}) and rotating
flows ({\red CITE PAPERS HERE}). Solving problems in fluid dynamics becomes
more difficult when multiple dynamics are included in the governing
equations. This is often due to the nonlinearity implicit to the Navier-Stokes
equation which prevents the superposition of solutions. For this reason, there
are fewer studies which involve both rotation and stratification, and among
those, few that are able to make general and comprehensive statements about rotating stratified
flows.

Despite the lack of analytical solutions of the Navier-Stokes equation
(coupled with the equations for mass conservation, an equation of state, and
other quantities), 


\section{Numerical methods for high resolution fluid dynamics}
\label{sec:nummethods}
\section{Multiscale theory for rotating and/or stratified flows}
\label{sec:multiscale}

\section{Instabilities and turbulence in rotating and/or stratified
flows}
\label{sec:instabilitesturbulence}
The topic of instabilities and turbulence in fluid dynamics is perhaps one of
the oldest problems. 

{\red talk about early investigation into instabilities}

For non-stratified flows, typical exhibitions of turbulence
are isotropic and generally follow the scaling laws posited by
Kolmogorov \citet{Kolmogorov1941a, Kolmogorov1941b, Kolmogorov1941c} which were
obtained using dimensional arguments. These
scaling laws include the Kolmogorov velocity $u_{\eta}$, time $\tau_{\eta}$ and
length $\eta$ scales, i.e.
 the unit velocity, time, and length scales at which viscous diffusion occurs. These quantities
 are given by $u_{\eta} = (\nu\varepsilon)^{1/4}$, $\tau_{\eta} =
 \sqrt{\nu/\varepsilon}$, and $\eta = (\nu^3/\varepsilon)^{1/4}$, where $\varepsilon$ is the average rate of
turbulent energy dissipation and $\nu$ is the kinematic viscosity. A Reynolds
number can be defined using these values $Re = u_{\eta}\eta/\nu$ and is equal to
one at the kolmogorov scale, implying that viscous diffusion is of $O(1)$ in the
Navier-Stokes equation. This foundation of isotropic turbulence has been
paramount in the development of models for viscous diffusion. These scaling laws
are valid only when the turbulence is isotropic. When, for example,
buoyancy forces create pancake vorticies in the flow as is typical in stratified
flows, the nature of these eddies is anisotropic. \cite{Ozmidov1965} used a dimensional
argument to show that there is a length scale $l_O = \sqrt{\varepsilon/N^3}$
below which isotropic turbulence is expected (where $N$ is the buoyancy
frequency of the flow). {\red Futhermore, there is a transition in the energy spectrum
once the wavenumber of the flow increases past $l_O^{-1}$. Typical isotropic
turbulence exhibits an energy spectrum proportional to $k^{-5/3}$. In a
stratified fluid, this energy spectrum above the Ozmidov scale exhibits an
energy spectrum proportional to $k_h^{-3}$ and then transitions to the kolmogorov
scaling ($k^{-5/3}$) once the flow becomes isotropic.}

\subsection{Non-rotating Stratified flows}

The zigzag instability first studied by
\citet{BillantChomaz2000a,BillantChomaz2000b,BillantChomaz2000c,
BillantChomaz2001} is perhaps one of the most important instabilities in
stratified flows. It has been shown by
\citet{BillantChomaz2000c,Hattori_Suzuki_Hirota_Khandelwal_2021,Guo_Taylor_Zhou_2024} {\red see if
other papers talk about the growth rate of the zigzag instability} that the
zigzag instability (also referenced as the mixed-hyperbolic instability) is
consistently the fastest growing instability mode for a stratified vortex column
(or dipole) for sufficient stratification. The most notable characteristic of
this instability is that it is a reliable mechanism for vertically invariant base flows
to gain vertical structure. Most notably, several studies have shown that the
vertical length scale of the zigzag instability scales with the Froude number and
therefore $k_z \propto Fr^{-1}$. A multiscale asymototic analysis of the
governing equations for strongly stratified flows recovers this scaling for the
vertical length scale and additionally finds that the turbulent horizontal
velocities scale with $Fr^{-1/2}$ and the turbulent vertical velocity scales with
$Fr^{-1}$ \citep{Chinial2022}. This result is valid in the
thermally-nondiffusive regime, i.e. when the P\'eclet number is greater than
$O(1)$, and alternate scaling laws are valid in the thermally diffusive regime
shown by \citet{Shah2023}. 

\subsection{Rotating Non-stratified flows}

\subsection{Rotating stratified flows}



\section{Research Proposal}
\label{sec:researchproposal}

    \begin{itemize}
        \item Numerically simulate rotating stratified flows with stochastic
        forcing
        \item Quantity mixing via modal analysis and instability structures
        \item Eventually propose multiscale asymptotic model for rotating
        stratified turbulence (find scaling laws for vertical velocity)
        \item Try different domains
    \end{itemize}

\bibliographystyle{jfm}
\bibliography{BuhlThesis}

\end{document}
