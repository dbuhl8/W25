
\documentclass[lineno]{jfm}

\usepackage{graphicx}
\usepackage{newtxtext}
\usepackage{newtxmath}
%\usepackage{amsmath}
%\usepackage{amsthm} %proof environment
%\usepackage{amssymb}
%\usepackage{amsfonts}
%\usepackage{enumitem} %nice lists
%\usepackage{verbatim} %useful for something 
\usepackage{xcolor}
%\usepackage{setspace}
%\usepackage{blindtext} % I have no idea what this is 
%\usepackage{caption}  % need this for unnumbered captions/figures
%\usepackage{tikz}
%\usepackage{soul} % need this for the hl command
\usepackage{natbib}
\usepackage{hyperref}
\hypersetup{
    colorlinks = true,
    urlcolor   = blue,
    citecolor  = black,
}
\newtheorem{lemma}{Lemma}
\newtheorem{corollary}{Corollary}
\newcommand{\RomanNumeralCaps}[1]
\linenumbers

\title{Precandidacy Exam: Report}

\author{Dante Buhl\aff{1}
  \corresp{\email{dbuhl@ucsc.edu}}}

\affiliation{\aff{1} Department of Applied Mathematics, Baskin School of
Engineering, Univeristy of California
Santa Cruz, 1156 High Street, Santa Cruz, CA, 95064 US}

\begin{document}

\newcommand{\red}{\color{red}}
\newcommand{\wrms}{w_{\text{rms}}}
\newcommand{\bs}[1]{\boldsymbol{#1}}
\newcommand{\tb}[1]{\textbf{#1}}
\newcommand{\bmp}[1]{\begin{minipage}{#1\textwidth}}
\newcommand{\emp}{\end{minipage}}
\newcommand{\R}{\mathbb{R}}
\newcommand{\C}{\mathbb{C}}
\newcommand{\N}{\mathcal{N}}
\newcommand{\K}{\bs{\mathrm{K}}}
\newcommand{\m}{\bs{\mu}_*}
\newcommand{\s}{\bs{\Sigma}_*}
\newcommand{\dt}{\Delta t}
\newcommand{\dx}{\Delta x}
\newcommand{\tr}[1]{\text{Tr}(#1)}
\newcommand{\Tr}[1]{\text{Tr}(#1)}
\newcommand{\Div}{\nabla \cdot}
\renewcommand{\div}{\nabla \cdot}
\newcommand{\Curl}{\nabla \times}
\newcommand{\Grad}{\nabla}
\newcommand{\grad}{\nabla}
\newcommand{\grads}{\nabla_s}
\newcommand{\gradf}{\nabla_f}
\newcommand{\xs}{\bs{x}_s}
\newcommand{\xf}{\bs{x}_f}
\newcommand{\ts}{t_s}
\newcommand{\tf}{t_f}
\newcommand{\pt}{\partial t}
\newcommand{\pz}{\partial z}
\newcommand{\uvec}{\bs{u}}
\newcommand{\F}{\bs{F}}
\newcommand{\T}{\tilde{T}}
\newcommand{\ez}{\bs{e}_z}
\newcommand{\ex}{\bs{e}_x}
\newcommand{\ey}{\bs{e}_y}
\newcommand{\eo}{\bs{e}_{\bs{\Omega}}}
\newcommand{\ppt}[1]{\frac{\partial #1}{\partial t}}
\newcommand{\ppts}[1]{\frac{\partial #1}{\partial t_s}}
\newcommand{\pptf}[1]{\frac{\partial #1}{\partial t_f}}
\newcommand{\ppz}[1]{\frac{\partial #1}{\partial z}}
\newcommand{\ddz}[1]{\frac{d #1}{d z}}
\newcommand{\ppzetas}[1]{\frac{\partial^2 #1}{\partial \zeta^2}}
\newcommand{\ppzs}[1]{\frac{\partial #1}{\partial z_s}}
\newcommand{\ppzf}[1]{\frac{\partial #1}{\partial z_f}}
\newcommand{\ppx}[1]{\frac{\partial #1}{\partial x}}
\newcommand{\ppy}[1]{\frac{\partial #1}{\partial y}}
\newcommand{\ppzeta}[1]{\frac{\partial #1}{\partial \zeta}}

\maketitle

%\begin{abstract}
    %stuff here
%\end{abstract}

%\begin{keywords}
    %stuff here
%\end{keywords}

\section{Introduction}
\label{sec:intro}

In the recent history of fluid dynamics, few studies have completely acclimated
the dynamics of both stratification and rotation in a fully comprehensive
manner. By contrast, many achievements have been made in studying isolated
dynamics, whereby the effects of other physical mechanisms are ignored in order to
better understand the instabilities, flow structures, and other properties which arise
under specific conditions. This has led to many discoveries about stratified flows ({\red CITE PAPERS HERE}) and rotating
flows ({\red CITE PAPERS HERE}). Solving problems in fluid dynamics becomes
mouch more difficult when multiple dynamics are included in the governing
equations. This is often due to the nonlinearity implicit to the Navier-Stokes
equation which prevents the superposition of solutions. For this reason, there
are fewer studies which involve both rotation and stratification, and among
those, few that are able to make general statements about rotating stratified
flows.

Despite the limitations to analytical solutions of the Navier-Stokes equation
(coupled with the equations for mass conservation, an equation of state, and
other quantities), 


\section{Numerical methods for high resolution fluid dynamics}
\label{sec:nummethods}
\section{Multiscale theory for rotating and/or stratified flows}
\label{sec:multiscale}

\section{Instabilities and turbulence in rotating and/or stratified
flows}
\label{sec:instabilitesturbulence}

\section{Research Proposal}
\label{sec:researchproposal}
keep this in here until I have a lot of citations going \cite{Chinial2022}

\bibliographystyle{jfm}
\bibliography{BuhlThesis}

\end{document}
