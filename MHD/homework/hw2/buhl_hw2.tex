\documentclass{article}

\usepackage{graphicx} % Required for inserting images
\usepackage[left=1in,right=1in,top=1in,bottom=1in]{geometry} \usepackage{amsmath}
\usepackage{amsthm} %proof environment
\usepackage{amsthm} %proof environment
\usepackage{amssymb}
\usepackage{amsfonts}
\usepackage{enumitem} %nice lists
\usepackage{verbatim} %useful for something 
\usepackage{xcolor}
\usepackage{setspace}
\usepackage{titlesec}
\usepackage{blindtext} % I have no idea what this is 
\usepackage{caption}  % need this for unnumbered captions/figures
\usepackage{natbib}
\usepackage{tikz}
\usepackage{hyperref}

\titleformat{\section}{\bfseries\Large}{Problem \thesection:}{5pt}{}

\begin{document}

\title{AM 275 - Magnetohydrodynamics:}
\author{Dante Buhl}


\newcommand{\wrms}{w_{\text{rms}}}
\newcommand{\bs}[1]{\boldsymbol{#1}}
\newcommand{\tb}[1]{\textbf{#1}}
\newcommand{\bmp}[1]{\begin{minipage}{#1\textwidth}}
\newcommand{\emp}{\end{minipage}}
\newcommand{\R}{\mathbb{R}}
\newcommand{\C}{\mathbb{C}}
\newcommand{\N}{\mathcal{N}}
\newcommand{\K}{\bs{\mathrm{K}}}
\newcommand{\m}{\bs{\mu}_*}
\newcommand{\s}{\bs{\Sigma}_*}
\newcommand{\dt}{\Delta t}
\newcommand{\dx}{\Delta x}
\newcommand{\tr}[1]{\text{Tr}(#1)}
\newcommand{\Tr}[1]{\text{Tr}(#1)}
\newcommand{\Div}{\nabla \cdot}
\renewcommand{\div}{\nabla \cdot}
\newcommand{\Curl}{\nabla \times}
\newcommand{\Grad}{\nabla}
\newcommand{\grad}{\nabla}
\newcommand{\grads}{\nabla_s}
\newcommand{\gradf}{\nabla_f}
\newcommand{\xs}{x_s}
\newcommand{\x}{\bs{x}}
\newcommand{\xf}{x_f}
\newcommand{\ts}{t_s}
\newcommand{\tf}{t_f}
\newcommand{\pt}{\partial t}
\newcommand{\pz}{\partial z}
\newcommand{\uvec}{\bs{u}}
\newcommand{\bvec}{\bs{B}}
\newcommand{\nvec}{\hat{\bs{n}}}
\newcommand{\B}{\bs{B}}
\newcommand{\A}{\bs{A}}
\newcommand{\jvec}{\bs{j}}
\newcommand{\F}{\bs{F}}
\newcommand{\T}{\tilde{T}}
\newcommand{\ez}{\bs{e}_z}
\newcommand{\ex}{\bs{e}_x}
\newcommand{\ey}{\bs{e}_y}
\newcommand{\eo}{\bs{e}_{\bs{\Omega}}}
\newcommand{\ppt}[1]{\frac{\partial #1}{\partial t}}
\newcommand{\pp}[2]{\frac{\partial #1}{\partial #2}}
\newcommand{\pptwo}[2]{\frac{\partial^2 #1}{\partial #2^2}}
\newcommand{\DDt}[1]{\frac{D #1}{D t}}
\newcommand{\ppts}[1]{\frac{\partial #1}{\partial t_s}}
\newcommand{\pptf}[1]{\frac{\partial #1}{\partial t_f}}
\newcommand{\ppz}[1]{\frac{\partial #1}{\partial z}}
\newcommand{\ddz}[1]{\frac{d #1}{d z}}
\newcommand{\ppzetas}[1]{\frac{\partial^2 #1}{\partial \zeta^2}}
\newcommand{\ppzs}[1]{\frac{\partial #1}{\partial z_s}}
\newcommand{\ppzf}[1]{\frac{\partial #1}{\partial z_f}}
\newcommand{\ppx}[1]{\frac{\partial #1}{\partial x}}
\newcommand{\ppxi}[1]{\frac{\partial #1}{\partial x_i}}
\newcommand{\ppxj}[1]{\frac{\partial #1}{\partial x_j}}
\newcommand{\ppy}[1]{\frac{\partial #1}{\partial y}}
\newcommand{\ppzeta}[1]{\frac{\partial #1}{\partial \zeta}}


\maketitle 
% This line removes the automatic indentation on new paragraphs
\setlength{\parindent}{0pt}

\section{Show that magnetic helicity is conserved}
\begin{proof}
    In order to show that the magnetic helicity is preserved over a material
    volume with a bounding surface with $\B\cdot\nvec = 0$ everywhere on the
    surface. We assume that the field lines (as well as the material volume) are
    frozen into the flow. This proof will follow the proof in the Davidson text. 
    Let us consider the time evolution of the magnetic helicity,
    \begin{align*}
        \ppt{h_m} &= \ppt{}\int_V \A\cdot\B dV\\
        &= \int_V \ppt{}(\A\cdot\B) + \grad\cdot(\A\cdot\B)\uvec dV\\
        &= \int_V \A\cdot\ppt{\B} + \B\cdot\ppt{A} + \grad\cdot(\A\cdot\B)\uvec
        dV\\
        &= \int_V \A\cdot\grad\times(\uvec\times\B) + \B\cdot\grad\phi +
        \grad\cdot(\A\cdot\B)\uvec dV
    \end{align*}
    Here the Davidson text very quickly rewrites what remains from the induction
    equation. I will show this derivation in a little more detail. 
    \begin{align*}
        \A\cdot(\grad\times(\uvec\times\B)) =  \A\cdot(\grad\times\bs{F}) &=
        \bs{F}\cdot\grad\times\A - \grad\cdot(\A\times\bs{F}) \\
        &= (\uvec\times\B)\times\B - \grad\cdot(\A\times\uvec\times\B)\\
        &=  -\grad\cdot(\A\times(\uvec\times\B))\\
        &=  -\grad\cdot((\A\cdot\B)\uvec - (\A\cdot\uvec)\B)\\
        &= -\grad \cdot (\A\cdot\B)\uvec + \grad\cdot(\A\cdot\uvec)\B
    \end{align*}
    In addition to the following identity, $\B\cdot\grad\phi =
    \grad\cdot(\phi\B) - \phi(\div{\B})$, 
    we find that this now reduces our original integral to the following , 
    \begin{align*}
        \ppt{h_m} &= \int_V \left(-\grad \cdot (\A\cdot\B)\uvec +
        \grad\cdot(\A\cdot\uvec)\B\right)
        + \grad\cdot(\phi\B) +
        \grad\cdot(\A\cdot\B)\uvec dV\\
        &= \int_V \grad\cdot(\phi + \A\cdot\uvec)\B dV
    \end{align*}
    The final step is to use the divergence theorem to write this as a surface
    integral. We then use fact that we have chosen a frozen into the flow
    field preserves the fact that $\B\cdot\nvec = 0$ everywhere along the
    boundary in time to state that the surface integral must be zero. That is,
    magnetic helicity is preserved in these conditions: over a material volume
    whose bounding surface is composed entirely of magnetic field lines and both
    the material volume and field lines are frozen into the flow. 
\end{proof}

\section{Solve using Cauchy solutions}
\begin{enumerate}
    \item $\uvec = \left< \sin(z), \cos(z), 0\right> , \quad \B(\x,0) = \left< y,
    z, x\right>$
    
    We begin with the Cauchy solution. 
    \begin{gather*}
        \x(\bs{a}, 0) = \left(a_1, a_2, a_3\right)\\
        \ppt{\x} = \left(\sin(z), \cos(z), 0\right)\\
        \x(\bs{a},t) = \left(a_1 + t\sin(a_3), a_2 + t\cos(a_3), a_3\right)\\
        \B(\bs{a},t) = \pp{\x}{\bs{a}}\B(\bs{a},0)
        = \left[\begin{array}{c c c}
        1 & 0 & t\cos(a_3)\\
        0 & 1 & -t\sin(s_3)\\
        0 & 0 & 1\end{array}\right] \left[\begin{array}{c}
        a_2\\a_3\\a_1\end{array}\right] = \left(a_2 + a_1t\cos(a_3), a_3 -
        a_1t\sin(a_3), a_1\right)
    \end{gather*}

    \item $\uvec = \left< \sin(z), \cos(z), 1\right>, \quad \B(\x,0) = \left< 1,
    1, 1\right>$
    \begin{gather*}
        \x(\bs{a}, 0) = \left(a_1, a_2, a_3\right)\\
        \ppt{\x} = \left(\sin(z), \cos(z), 1\right)\\
        \x(\bs{a},t) = \left(a_1 + t\sin(a_3 + t), a_2 + t\cos(a_3 + t), a_3 + t\right)\\
        \B(\bs{a},t) = \pp{\x}{\bs{a}}\B(\bs{a},0)
        = \left[\begin{array}{c c c}
        1 & 0 & t\cos(a_3)\\
        0 & 1 & -t\sin(s_3)\\
        0 & 0 & 1\end{array}\right] \left[\begin{array}{c}
        1\\1\\1\end{array}\right] = \left(1 + t\cos(a_3+t), 1 -
        t\sin(a_3+t), 1\right)
    \end{gather*}
    \item $\uvec = \left< \sin(z), \cos(z), 1\right>, \quad \B(\x,0) = \left< x,
    y, -2z\right>$
    \begin{gather*}
        \x(\bs{a}, 0) = \left(a_1, a_2, a_3\right)\\
        \ppt{\x} = \left(\sin(z), \cos(z), 1\right)\\
        \x(\bs{a},t) = \left(a_1 + t\sin(a_3 + t), a_2 + t\cos(a_3 + t), a_3 + t\right)\\
        \B(\bs{a},t) = \pp{\x}{\bs{a}}\B(\bs{a},0)
        = \left[\begin{array}{c c c}
        1 & 0 & t\cos(a_3)\\
        0 & 1 & -t\sin(s_3)\\
        0 & 0 & 1\end{array}\right] \left[\begin{array}{c}
        a_1\\a_2\\-2a_3\end{array}\right] = \left(a_1 - 2a_3t\cos(a_3+t), a_2 +
        2a_3t\sin(a_3+t), -2a_3\right)
    \end{gather*}
    \item $\uvec = \left< y, -x, 0\right>, \quad \B(\x,0) = \left< x,
    -y, 0\right>$
    \begin{gather*}
        \x(\bs{a}, 0) = \left(a_1, a_2, a_3\right)\\
        \ppt{\x} = \left(y, -x, 0\right)\\
        x' = -y, \quad y' = x\\
        \frac{d^2x}{dt} = -x, \quad \frac{d^2y}{dt} = -y\\
        x(t) = x_1\cos(t) + x_2\sin(t), \quad y(t) = y_1\cos(t) + y_2\sin(t)\\
        x(0) = a_1, \quad x'(0) = -a_2, \quad y(0) = a_2, \quad y'(0) = a_1\\
        x_1 = a_1, \quad x_2 = -a_2, \quad y_1 = a_2, \quad y_2 = a_1\\
        \x(\bs{a},t) = \left(a_1\cos(t) - a_2\sin(t), a_2\cos(t) + a_1\sin(t),
        a_3\right)\\
        \B(\bs{a},t) = \pp{\x}{\bs{a}}\B(\bs{a},0)
        = \left[\begin{array}{c c c}
        \cos(t) & -\sin(t) & 0\\
        \sin(t) & \cos(t) & 0\\
        0 & 0 & 1\end{array}\right] \left[\begin{array}{c}
        a_1\\-a_2\\0\end{array}\right] = \left(a_1\cos(t) + a_2\sin(t),
        a_1\sin(t) - a_2\cos(t), 0\right)
    \end{gather*}
\end{enumerate}

\end{document}
