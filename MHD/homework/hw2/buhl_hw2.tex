\documentclass{article}

\usepackage{graphicx} % Required for inserting images
\usepackage[left=1in,right=1in,top=1in,bottom=1in]{geometry} \usepackage{amsmath}
\usepackage{amsthm} %proof environment
\usepackage{amssymb}
\usepackage{amsfonts}
\usepackage{enumitem} %nice lists
\usepackage{verbatim} %useful for something 
\usepackage{xcolor}
\usepackage{setspace}
\usepackage{titlesec}
\usepackage{blindtext} % I have no idea what this is 
\usepackage{caption}  % need this for unnumbered captions/figures
\usepackage{natbib}
\usepackage{tikz}
\usepackage{hyperref}

\titleformat{\section}{\bfseries\Large}{Problem \thesection:}{5pt}{}

\begin{document}

\title{AM 275 - Magnetohydrodynamics:}
\author{Dante Buhl}


\newcommand{\wrms}{w_{\text{rms}}}
\newcommand{\bs}[1]{\boldsymbol{#1}}
\newcommand{\tb}[1]{\textbf{#1}}
\newcommand{\bmp}[1]{\begin{minipage}{#1\textwidth}}
\newcommand{\emp}{\end{minipage}}
\newcommand{\R}{\mathbb{R}}
\newcommand{\C}{\mathbb{C}}
\newcommand{\N}{\mathcal{N}}
\newcommand{\K}{\bs{\mathrm{K}}}
\newcommand{\m}{\bs{\mu}_*}
\newcommand{\s}{\bs{\Sigma}_*}
\newcommand{\dt}{\Delta t}
\newcommand{\dx}{\Delta x}
\newcommand{\tr}[1]{\text{Tr}(#1)}
\newcommand{\Tr}[1]{\text{Tr}(#1)}
\newcommand{\Div}{\nabla \cdot}
\renewcommand{\div}{\nabla \cdot}
\newcommand{\Curl}{\nabla \times}
\newcommand{\Grad}{\nabla}
\newcommand{\grad}{\nabla}
\newcommand{\grads}{\nabla_s}
\newcommand{\gradf}{\nabla_f}
\newcommand{\xs}{x_s}
\newcommand{\xf}{x_f}
\newcommand{\ts}{t_s}
\newcommand{\tf}{t_f}
\newcommand{\pt}{\partial t}
\newcommand{\pz}{\partial z}
\newcommand{\uvec}{\bs{u}}
\newcommand{\bvec}{\bs{B}}
\newcommand{\nvec}{\hat{\bs{n}}}
\newcommand{\B}{\bs{B}}
\newcommand{\A}{\bs{A}}
\newcommand{\jvec}{\bs{j}}
\newcommand{\F}{\bs{F}}
\newcommand{\T}{\tilde{T}}
\newcommand{\ez}{\bs{e}_z}
\newcommand{\ex}{\bs{e}_x}
\newcommand{\ey}{\bs{e}_y}
\newcommand{\eo}{\bs{e}_{\bs{\Omega}}}
\newcommand{\ppt}[1]{\frac{\partial #1}{\partial t}}
\newcommand{\pp}[2]{\frac{\partial #1}{\partial #2}}
\newcommand{\pptwo}[2]{\frac{\partial^2 #1}{\partial #2^2}}
\newcommand{\DDt}[1]{\frac{D #1}{D t}}
\newcommand{\ppts}[1]{\frac{\partial #1}{\partial t_s}}
\newcommand{\pptf}[1]{\frac{\partial #1}{\partial t_f}}
\newcommand{\ppz}[1]{\frac{\partial #1}{\partial z}}
\newcommand{\ddz}[1]{\frac{d #1}{d z}}
\newcommand{\ppzetas}[1]{\frac{\partial^2 #1}{\partial \zeta^2}}
\newcommand{\ppzs}[1]{\frac{\partial #1}{\partial z_s}}
\newcommand{\ppzf}[1]{\frac{\partial #1}{\partial z_f}}
\newcommand{\ppx}[1]{\frac{\partial #1}{\partial x}}
\newcommand{\ppxi}[1]{\frac{\partial #1}{\partial x_i}}
\newcommand{\ppxj}[1]{\frac{\partial #1}{\partial x_j}}
\newcommand{\ppy}[1]{\frac{\partial #1}{\partial y}}
\newcommand{\ppzeta}[1]{\frac{\partial #1}{\partial \zeta}}


\maketitle 
% This line removes the automatic indentation on new paragraphs
\setlength{\parindent}{0pt}

\section{Show that magnetic helicity if conserved}

\begin{proof}
    Show that the following quantity is conserved with time. 
    \begin{gather*}
        h_m = \int_V \A\cdot\B dV, \quad \B = \grad\times\A\\
        \ppt{h_m} = \ppt{}\left(\int_V \A\cdot\B dV\right)\\
        \ppt{h_m} = \int_V \ppt{}\left(\A\cdot\B\right) dV + \int_{\partial V}
        (\A\cdot\B)(\uvec\cdot\nvec) dS
    \end{gather*}
    where $V$ is a closed volume in the domain of the fluid and $\partial V$ is
    the boundary surface of that closed volume. According to the Reynolds
    transport theorem we have the following time derivative of the magnetic
    helicity. 
    We can follow up by taking the time derivatives of $\A\cdot\B$. 
    \begin{align*}
        \ppt{h_m} &= \int_V \ppt{}\left(\A\cdot\B\right) dV + \int_{\partial V}
        (\A\cdot\B)(\uvec\cdot\nvec) dS\\
        &= \int_V \ppt{\A}\cdot\B + \A\cdot\ppt{\B} dV + \int_{\partial V}
        (\A\cdot\B)(\uvec\cdot\nvec) dS\\
        &= \int_V \B\cdot\grad\phi + \A\cdot\left(-(\uvec\cdot\grad)\B +
        (\B\cdot\grad)\uvec + \B(\div{\uvec})\right) + \int_{\partial V}
        (\A\cdot\B)(\uvec\cdot\nvec) dS\\
        &= \int_V \B\cdot\grad\phi + \A\cdot\left(-(\uvec\cdot\grad)\B +
        (\B\cdot\grad)\uvec + \B(\div{\uvec})\right) + (\grad\cdot
        (\A\cdot\B)\uvec) dV\\
        &= \int_V \B\cdot\grad\phi -\A\cdot(\uvec\cdot\grad)\B +
        \A\cdot(\B\cdot\grad)\uvec + (\A\cdot\B)(\div{\uvec})\right) + (\grad\cdot
        (\A\cdot\B)\uvec) dV\\
        &= \int_V \B\cdot\grad\phi -\A\cdot(\uvec\cdot\grad)\B +
        \A\cdot(\B\cdot\grad)\uvec - \uvec\cdot\grad(\A\cdot\B) dV\\
        &= \int_V \B\cdot\grad\phi -\B\cdot(\uvec\cdot\grad)\A +
        \A\cdot(\B\cdot\grad)\uvec  dV\\
    \end{align*}
    where we use $\ppt{\A} = \uvec\times\B + \grad\phi$ from the lecture note
    and the Davidson text. 


\end{proof}

\section{Solve using Cauchy solutions}


\end{document}
