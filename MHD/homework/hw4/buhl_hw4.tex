\documentclass{article}

\usepackage{graphicx} % Required for inserting images
\usepackage[left=1in,right=1in,top=1in,bottom=1in]{geometry} \usepackage{amsmath}
\usepackage{amsthm} %proof environment
\usepackage{amsthm} %proof environment
\usepackage{amssymb}
\usepackage{amsfonts}
\usepackage{enumitem} %nice lists
\usepackage{verbatim} %useful for something 
\usepackage{xcolor}
\usepackage{setspace}
\usepackage{titlesec}
\usepackage{blindtext} % I have no idea what this is 
\usepackage{caption}  % need this for unnumbered captions/figures
\usepackage{natbib}
\usepackage{tikz}
\usepackage{hyperref}

\titleformat{\section}{\bfseries\Large}{Problem \thesection:}{5pt}{}

\begin{document}

\title{AM 275 - Magnetohydrodynamics: Homework 4}
\author{Dante Buhl}


\newcommand{\wrms}{w_{\text{rms}}}
\newcommand{\bs}[1]{\boldsymbol{#1}}
\newcommand{\tb}[1]{\textbf{#1}}
\newcommand{\bmp}[1]{\begin{minipage}{#1\textwidth}}
\newcommand{\emp}{\end{minipage}}
\newcommand{\R}{\mathbb{R}}
\newcommand{\C}{\mathbb{C}}
\newcommand{\N}{\mathcal{N}}
%\newcommand{\K}{\bs{\mathrm{K}}}
\newcommand{\m}{\bs{\mu}_*}
\newcommand{\s}{\bs{\Sigma}_*}
\newcommand{\dt}{\Delta t}
\newcommand{\dx}{\Delta x}
\newcommand{\tr}[1]{\text{Tr}(#1)}
\newcommand{\Tr}[1]{\text{Tr}(#1)}
\newcommand{\Div}{\nabla \cdot}
\renewcommand{\div}{\nabla \cdot}
\newcommand{\Curl}{\nabla \times}
\newcommand{\Grad}{\nabla}
\newcommand{\grad}{\nabla}
\newcommand{\grads}{\nabla_s}
\newcommand{\gradf}{\nabla_f}
\newcommand{\xs}{x_s}
\newcommand{\x}{\bs{x}}
\newcommand{\xf}{x_f}
\newcommand{\ts}{t_s}
\newcommand{\tf}{t_f}
\newcommand{\pt}{\partial t}
\newcommand{\pz}{\partial z}
\newcommand{\uvec}{\bs{u}}
\newcommand{\bvec}{\bs{B}}
\newcommand{\nvec}{\hat{\bs{n}}}
\newcommand{\B}{\bs{B}}
\newcommand{\A}{\bs{A}}
\newcommand{\jvec}{\bs{j}}
\newcommand{\F}{\bs{F}}
\newcommand{\T}{\tilde{T}}
\newcommand{\ez}{\bs{e}_z}
\newcommand{\ex}{\bs{e}_x}
\newcommand{\ey}{\bs{e}_y}
\newcommand{\eo}{\bs{e}_{\bs{\Omega}}}
\newcommand{\ppt}[1]{\frac{\partial #1}{\partial t}}
\newcommand{\pp}[2]{\frac{\partial #1}{\partial #2}}
\newcommand{\pptwo}[2]{\frac{\partial^2 #1}{\partial #2^2}}
\newcommand{\ddtwo}[2]{\frac{d^2 #1}{d #2^2}}
\newcommand{\DDt}[1]{\frac{D #1}{D t}}
\newcommand{\ppts}[1]{\frac{\partial #1}{\partial t_s}}
\newcommand{\pptf}[1]{\frac{\partial #1}{\partial t_f}}
\newcommand{\ppz}[1]{\frac{\partial #1}{\partial z}}
\newcommand{\ddz}[1]{\frac{d #1}{d z}}
\newcommand{\ppzetas}[1]{\frac{\partial^2 #1}{\partial \zeta^2}}
\newcommand{\ppzs}[1]{\frac{\partial #1}{\partial z_s}}
\newcommand{\ppzf}[1]{\frac{\partial #1}{\partial z_f}}
\newcommand{\ppx}[1]{\frac{\partial #1}{\partial x}}
\newcommand{\ddx}[1]{\frac{d #1}{d x}}
\newcommand{\ppxi}[1]{\frac{\partial #1}{\partial x_i}}
\newcommand{\ppxj}[1]{\frac{\partial #1}{\partial x_j}}
\newcommand{\ppy}[1]{\frac{\partial #1}{\partial y}}
\newcommand{\ppzeta}[1]{\frac{\partial #1}{\partial \zeta}}
\renewcommand{\k}{\bs{k}}
\newcommand{\real}[1]{\text{Re}\left[#1\right]}


\maketitle 
% This line removes the automatic indentation on new paragraphs
\setlength{\parindent}{0pt}

\section{}
\begin{enumerate}[label=\alph*.)]
    \item We begin here with the linearized perturbation equations taken from
    the lecture, but now including the Coriolis force. We will ignore the
    centripedal force here as claim that it acts only on the base flow and not
    the perturbations. {\color{red} make a better excuse}. 
    \begin{gather*}
        \rho_0\left(\ppt{\uvec'} \right) + 2\rho_0\bs{\Omega}\times\uvec'= -\grad \left(p' +
        \frac{1}{\mu_0}\B'\cdot\overline{\B}\right) + \frac{1}{\mu_0}(\overline{\B}\cdot\grad)\B', \quad \div{\uvec'}
        = 0\\
        \ppt{\B'} = (\overline{\B}\cdot\grad)\uvec', \quad \div{\B'} = 0
    \end{gather*}
    From these equations, we proceed with the usual wave-like ansatz whereby
    each quantity (vector and scalar) is of the form, 
    \begin{gather*}
        q' = \tilde{q}e^{i\left(\k\cdot\x - \omega t\right)}
    \end{gather*}
    where $\tilde{q}$ is generally a complex scalar (or vector) and the
    exponential term follows Euler's identity. We give specific attention to the
    complex components of any general $q'$. We have very specifically, that
    conditions on that quantity (for example, the divergence of $\uvec'$ and boundary
    conditions on the relevant characteristics of the flow) are only considered
    in real space. That is, we would care very rigorously about the
    $\real{\div{\uvec}} = 0$ and not necessarily about the imaginary component
    (Note that in this specific case there is an extra i floating around and can
    be factored out so it happens that both the real and imaginary components of
    the divergence of $\uvec'$ must be zero). We have as a consequence that
    this simplifies these equations to an algebraic dilema rather than that of a
    partial differential equation. These equations then have the form, 
    \begin{gather*}
        -i\rho_0\omega\tilde{\uvec} + 2\rho_0\bs{\Omega}\times\tilde{\uvec} =
        -i\k\left(\tilde{p} + \frac{1}{\mu_0}\overline{\B}\cdot\tilde{\B}\right)
        + \frac{i}{\mu_0}\left(\overline{\B}\cdot\k\right)\tilde{\B}, \quad
        i\k\cdot\tilde{\uvec} = 0\\
        -i\omega\tilde{\B} = i(\overline{\B}\cdot\k)\tilde{\uvec}, \quad
        i\k\cdot\tilde{\B}
    \end{gather*}
    In order to proceed, we take a pseudo-curl in this vector space. Note that
    since we have taken $\nabla = i\k$, and thus we take $i\k\times(\cdot)$ for
    the momentum and induction equation. 
    \begin{gather*}
        \rho_0\omega(\k\times\tilde{\uvec}) +
        2i\rho_0(\k\times(\bs{\Omega}\times\tilde{\uvec})) =
        - \frac{1}{\mu_0}\left(\overline{\B}\cdot\k\right)(\k\times\tilde{\B})\\
        -i\omega(\k\times\tilde{\B}) = i(\overline{\B}\cdot\k)(\k\times\tilde{\uvec})
    \end{gather*}
    Here the ``curl'' of the Coriolis term is given specific attention (whereby
    the NRL plasma formulary 2019 is referenced) and we have,
    \begin{gather*}
        \rho_0\omega(\k\times\tilde{\uvec}) -
        2i\rho_0(\k\cdot\bs{\Omega})\tilde{\uvec} =
        - \frac{1}{\mu_0}\left(\overline{\B}\cdot\k\right)(\k\times\tilde{\B})\\
        \omega(\k\times\tilde{\B}) = -(\overline{\B}\cdot\k)(\k\times\tilde{\uvec})
    \end{gather*}
    Now, upon first inspection, we see something quite odd. We have (and this
    must be satisfied) that $\k\times\tilde{\uvec}$ must be colinear to
    $\tilde{\uvec}$. This first appears non-sensicle as multivariate calculus
    has taught us that the cross product of two vectors must be perpendicular to
    both those vectors. Here, we point to the consideration of these complex
    quantities representing real (physical) quantities at the end of the day. We
    can imagine different criterium for the orthogonality of these complex
    vectors, that is we must chose a specific inner product to satisfy
    orthogonality. Since we care about the real component of the inner product
    we must have that, 
    \begin{align*}
        \real{\left<\tilde{\uvec},\k\times\tilde{\uvec}\right>} &= 0\\
        \real{\left<\tilde{\uvec},\k\times\tilde{\uvec}\right>} &=
        \real{(\tilde{\uvec}_R^T -i\tilde{\uvec}_I^T) ((\k\times\tilde{\uvec}_R) +
        i(\k\times\tilde{\uvec}_I))}\\
        &= \real{i\tilde{\uvec}_R\cdot(\k\times\tilde{\uvec}_I)
        -i\tilde{\uvec}_I\cdot(\k\times\tilde{\uvec}_R)}\\
        &= 0
    \end{align*}
    and yet the magnitude/modulus/absolute-value of this cross product is
    non-zero!! It is for this precise reason that we have that $\tilde{\uvec}$
    can be colinear to $\k\times\tilde{\uvec}$ (note the i attached to
    $\tilde{\uvec}$ in the momentum / vorticity equation). Thus we proceed with
    the notion that we have some wiggle room in an initially seemingly obsurb
    statement. We continue with the substitution made in class. 
    \begin{gather*}
        \rho_0\omega(\k\times\tilde{\uvec}) -
        2i\rho_0(\k\cdot\bs{\Omega})\tilde{\uvec} =
        \frac{1}{\omega\mu_0}\left(\overline{\B}\cdot\k\right)^2(\k\times\tilde{\uvec})\\
        \rho_0\omega^2(\k\times\tilde{\uvec}) -
        2i\omega\rho_0(\k\cdot\bs{\Omega})\tilde{\uvec} -
        \frac{1}{\mu_0}\left(\overline{\B}\cdot\k\right)^2(\k\times\tilde{\uvec})
        = 0
    \end{gather*}
    Finally we take the inner product of this entire equation with
    $(\k\times\tilde{\uvec})/|\k\times\tilde{\uvec}|^2$. We have then, 
    \begin{gather*}
        \omega^2 -
        2\omega(\k\cdot\bs{\Omega})\frac{\left<i\tilde{\uvec},
        \k\times\tilde{\uvec}\right>}{|\k\times\tilde{\uvec}|^2} -
        \frac{1}{\rho_0\mu_0}\left(\overline{\B}\cdot\k\right)^2
        = 0
    \end{gather*}
    In order to resolve this equation, we simply need to resolve the two inner
    products. It simplifies to the following:

    And thus we have according to these equations, 
    \begin{gather*}
        \omega^2 -
        2\frac{\omega}{|\k|}(\k\cdot\bs{\Omega}) -
        \frac{1}{\rho_0\mu_0}\left(\overline{\B}\cdot\k\right)^2
        = 0
    \end{gather*}

\end{enumerate}

\end{document}
