\documentclass{article}

\usepackage{graphicx} % Required for inserting images
\usepackage[left=1in,right=1in,top=1in,bottom=1in]{geometry} \usepackage{amsmath}
\usepackage{amsthm} %proof environment
\usepackage{amsthm} %proof environment
\usepackage{amssymb}
\usepackage{amsfonts}
\usepackage{enumitem} %nice lists
\usepackage{verbatim} %useful for something 
\usepackage{xcolor}
\usepackage{setspace}
\usepackage{titlesec}
\usepackage{blindtext} % I have no idea what this is 
\usepackage{caption}  % need this for unnumbered captions/figures
\usepackage{natbib}
\usepackage{tikz}
\usepackage{hyperref}

\titleformat{\section}{\bfseries\Large}{Problem \thesection:}{5pt}{}

\begin{document}

\title{AM 275 - Magnetohydrodynamics:}
\author{Dante Buhl}


\newcommand{\wrms}{w_{\text{rms}}}
\newcommand{\bs}[1]{\boldsymbol{#1}}
\newcommand{\tb}[1]{\textbf{#1}}
\newcommand{\bmp}[1]{\begin{minipage}{#1\textwidth}}
\newcommand{\emp}{\end{minipage}}
\newcommand{\R}{\mathbb{R}}
\newcommand{\C}{\mathbb{C}}
\newcommand{\N}{\mathcal{N}}
\newcommand{\K}{\bs{\mathrm{K}}}
\newcommand{\m}{\bs{\mu}_*}
\newcommand{\s}{\bs{\Sigma}_*}
\newcommand{\dt}{\Delta t}
\newcommand{\dx}{\Delta x}
\newcommand{\tr}[1]{\text{Tr}(#1)}
\newcommand{\Tr}[1]{\text{Tr}(#1)}
\newcommand{\Div}{\nabla \cdot}
\renewcommand{\div}{\nabla \cdot}
\newcommand{\Curl}{\nabla \times}
\newcommand{\Grad}{\nabla}
\newcommand{\grad}{\nabla}
\newcommand{\grads}{\nabla_s}
\newcommand{\gradf}{\nabla_f}
\newcommand{\xs}{x_s}
\newcommand{\x}{\bs{x}}
\newcommand{\xf}{x_f}
\newcommand{\ts}{t_s}
\newcommand{\tf}{t_f}
\newcommand{\pt}{\partial t}
\newcommand{\pz}{\partial z}
\newcommand{\uvec}{\bs{u}}
\newcommand{\bvec}{\bs{B}}
\newcommand{\nvec}{\hat{\bs{n}}}
\newcommand{\B}{\bs{B}}
\newcommand{\A}{\bs{A}}
\newcommand{\jvec}{\bs{j}}
\newcommand{\F}{\bs{F}}
\newcommand{\T}{\tilde{T}}
\newcommand{\ez}{\bs{e}_z}
\newcommand{\ex}{\bs{e}_x}
\newcommand{\ey}{\bs{e}_y}
\newcommand{\eo}{\bs{e}_{\bs{\Omega}}}
\newcommand{\ppt}[1]{\frac{\partial #1}{\partial t}}
\newcommand{\pp}[2]{\frac{\partial #1}{\partial #2}}
\newcommand{\pptwo}[2]{\frac{\partial^2 #1}{\partial #2^2}}
\newcommand{\ddtwo}[2]{\frac{d^2 #1}{d #2^2}}
\newcommand{\DDt}[1]{\frac{D #1}{D t}}
\newcommand{\ppts}[1]{\frac{\partial #1}{\partial t_s}}
\newcommand{\pptf}[1]{\frac{\partial #1}{\partial t_f}}
\newcommand{\ppz}[1]{\frac{\partial #1}{\partial z}}
\newcommand{\ddz}[1]{\frac{d #1}{d z}}
\newcommand{\ppzetas}[1]{\frac{\partial^2 #1}{\partial \zeta^2}}
\newcommand{\ppzs}[1]{\frac{\partial #1}{\partial z_s}}
\newcommand{\ppzf}[1]{\frac{\partial #1}{\partial z_f}}
\newcommand{\ppx}[1]{\frac{\partial #1}{\partial x}}
\newcommand{\ddx}[1]{\frac{d #1}{d x}}
\newcommand{\ppxi}[1]{\frac{\partial #1}{\partial x_i}}
\newcommand{\ppxj}[1]{\frac{\partial #1}{\partial x_j}}
\newcommand{\ppy}[1]{\frac{\partial #1}{\partial y}}
\newcommand{\ppzeta}[1]{\frac{\partial #1}{\partial \zeta}}


\maketitle 
% This line removes the automatic indentation on new paragraphs
\setlength{\parindent}{0pt}

\section{}
\begin{enumerate}[label=\alph*.)]
    \item  We begin solving this problem by simply setting $\jvec\times\B = 0$. 
    \begin{gather*}
        \B = \left< B_x, B_y, 0\right>, \quad \jvec = \left< -\ppz{B_y},
        \ppz{B_x}, 0\right>\\
        \jvec\times\B = \left<0, 0, -B_y\ppz{B_y} - B_x\ppz{B_x}\right>\\
        \jvec\times\B = 0 \implies \frac{1}{2}\ppz{}\left(B_x^2 + B_y^2\right) =
        0 \implies B_x^2 + B_y^2 = \text{constant}
    \end{gather*}
    We have obtained the condition for this field to be force free, and in the
    proceeding problems it will become clear why this satisfies the Beltrami
    condition in Helmholtz form. 

    \item Next we enforce the Beltrami property, i.e. $\grad\times\B =
    \alpha\B$. We have, 
    \begin{gather*}
        \left< -\ppz{B_y}, \ppz{B_x}, 0\right> = \alpha\left<B_x, B_y,
        0\right>\\
        \alpha B_x = - \ppz{B_y}, \quad \alpha B_y = \ppz{B_x}
    \end{gather*}
    \item  We will now show using substitution that this field satisfies both
    the Beltrami property in the Helmholtz form as well as the force-free
    condition derived in part a. 
    \begin{gather*}
        \ppz{}\left(\alpha B_x = - \ppz{B_y}, \quad \alpha B_y =
        \ppz{B_x}\right)\\
        \alpha\ppx{B_x} = -\pptwo{B_y}{z}, \quad \alpha\ppz{B_y} =
        \pptwo{B_x}{z}\\
        \alpha^2 B_y = -\pptwo{B_y}{z}, \quad -\alpha^2 B_x =
        \pptwo{B_x}{z}
    \end{gather*}
    We note that since $B_x$ and $B_y$ depend only on $z$, this exactly
    satisfies the Helmholtz equation for the Beltrami property. We notice a
    standard form for $B_x$ and $B_y$ given by,
    \begin{gather*}
        B_x = a\cos(\alpha z) + b\sin(\alpha z), \quad  B_y = c\cos(\alpha z) +
        d\sin(\alpha z)
    \end{gather*}
    Notice that given the condition found in part b, that we have a limited
    choice for the coefficients $a, b, c, d$. Specifically, 
    \begin{gather*}
        \alpha\left(a\cos(\alpha z) + b\sin(\alpha z)\right) = -\alpha c\sin(\alpha z) +
        \alpha d\cos(\alpha z) \implies a = d, \quad b = -c\\
        B_x = d\cos(\alpha z) - c\sin(\alpha z), \quad  B_y = c\cos(\alpha z) +
        d\sin(\alpha z)\\
        B_x^2 + B_y^2 = d^2\cos^2(\alpha z) + c^2\sin^2(\alpha z) -
        2cd\cos(\alpha z)\sin(\alpha z) + c^2\cos^2(\alpha z) + d^2\sin^2(\alpha z) +
        2cd\cos(\alpha z)\sin(\alpha z)\\
         = d^2 + c^2 = \text{constant}
    \end{gather*}
\end{enumerate}

\section{}
    We begin this problem by writing the steady induction equation for an
    incompressible fluid. We have, 
    \begin{gather*}
        (\uvec\cdot\grad)\B = (\B\cdot\grad)\uvec + \eta\grad^2\B
    \end{gather*}
    For the given flow field $\uvec = U_0\left(-x, y, 0\right)$ and $\B =
    \left(0, B(x), 0\right)$, this reduces to,
    \begin{gather*}
        -U_0x\ppx{B} = U_0B + \eta\pptwo{B}{x}\\
        \eta\ddtwo{B}{x} + U_0x\ddx{B} + U_0B = 0
    \end{gather*}
    This ODE can be solved in two manners. First, and probably the most general
    argument is that we can write $U_0(x\ddx{B} + B) = U_0\ddx{}\left(xB\right)$ from which
    you can directly integrate and then use integration by parts to show the
    solution. The alternate method is to presume an ansatz for $B$ and then
    solve the resultant ODE obtained from that. We will do the latter, 
    \begin{gather*}
        B = e^{f(x)}, \quad B' = f'e^f, \quad B'' = f'^2e^f + f''e^f\\
        \eta\left(f'^2 + f''\right)e^f + U_0xf'e^f + U_0e^f = 0\\
        \eta f'^2 + \eta f'' + U_0xf' + U_0 = 0
    \end{gather*}
    Here we will attempt to solve using $f = Cx^{\alpha}$. This gives,
    \begin{gather*}
        \eta C^2\alpha^2x^{2\alpha - 2} + \eta C\alpha(\alpha -1)x^{\alpha - 2}
        + U_0C\alpha x^{\alpha} + U_0 = 0 
   \end{gather*}
    Notice that we have 4 different orders of x in this solution and yet it must
    be satisfied for all x. We must chose $\alpha$ such that there are actually
    only 2 different orders of $x$ and then solve for the coefficient $C$.
    Notice that $\alpha = 2$ yields $2\alpha - 2 = \alpha$ and $\alpha - 2 =
    0$. Thus we substitute $\alpha = 2$ into the ODE. 
    \begin{gather*}
        4\eta C^2 = -2U_0C, \quad 2\eta C = -U_0
    \end{gather*}
    These two equations are thankfully linearly dependent, i.e. thery are
    rescaled versions of one another. Thus we find $C = -U_0/2\eta$. Finally, we
    have,
    \begin{gather*}
        B(x) = B_0e^{-U_0x^2/2\eta}
    \end{gather*}

\section{}
    The solution of this problem derives first from expanding each term in the
    navier stokes equation from the presummed magnetohydrostatic balance. We
    have, 
    \begin{gather*}
        \B = \left(B_x, 0, B\right),\quad \jvec = \frac{1}{\mu_0}\left(0, -\ppx{B}, 0\right),
        \quad \jvec\times\B = -\frac{B}{\mu_0}\ddx{B}\ex + \frac{B_x}{\mu_0}\ddx{B} \ez\\
        p = \alpha\rho, \quad \grad p = \alpha\ddx{\rho}\ex\\
        -\alpha\ddx{\rho} = \frac{B}{\mu_0}\ddx{B}, \quad
        \frac{B_x}{\mu_0}\ddx{B} = \rho g
    \end{gather*}
    Notice here that we obtain a series of coupled ODEs for $B$ and $\rho$. More
    importantly, we can use a derivative identity to write, 
    \begin{gather*}
        \ddx{\rho} = -\frac{1}{2\alpha\mu_0}\ddx{B^2} \implies \rho =
        -\frac{B^2}{2\alpha\mu_0} + c
    \end{gather*}
    Using the boundary conditions given ($B(0) = 0$ and $\rho(0) = \rho_0$) we
    find that $c = \rho_0$. Now we substitute into the other obtained equation, 
    \begin{gather*}
        \frac{B_x}{\mu_0}\ddx{B} = -\frac{gB^2}{2\alpha\mu_0} + \rho_0g\\
        \ddx{B} = -\frac{gB^2}{2\alpha B_x} +
        \frac{\mu_0\rho_0g}{B_x}
    \end{gather*}
    Now, if one were to naively approach this ODE, they might have some trouble
    finding a solution. We, however, who are primed to find a solution of the
    form $\tanh(x)$, know that the derivative of $\tanh(x)$ is $1 - \tanh^2(x)$,
    and might notice some resemblance in the resultant ODE. We therefore write an
    ansatz of the form $B(x) = C\tanh(f(x))$. 
    \begin{gather*}
        B' = Cf'(1 - \tanh^2(f)) = Cf' - \frac{f'}{C}B^2\\
        \frac{\mu_0\rho_0g}{B_x} = Cf',\quad \frac{g}{2\alpha B_x} =
        \frac{f'}{C}\\
        \frac{\mu_0\rho_0g}{B_x} = \frac{C^2g}{2\alpha B_x} \implies C =
        \sqrt{2\mu_0\alpha\rho_0}\\
        f' = \frac{g\sqrt{\mu_0\rho_0}}{B_x\sqrt{2\alpha}} \implies f =
        \frac{g\sqrt{\mu_0\rho_0}}{B_x\sqrt{2\alpha}}x
    \end{gather*}
    Thus we have solved for $B$ to satisfy the obtained ODE and find in
    conclusion, 
    \begin{gather*}
        B(x) = \sqrt{2\mu_0\alpha\rho_0}\tanh\left(\sqrt{\frac{\mu_0\rho_0}{2\alpha}}
        \frac{gx}{B_x}\right)
    \end{gather*}


\end{document}
