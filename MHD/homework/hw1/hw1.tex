\documentclass{article}

\usepackage{graphicx} % Required for inserting images
\usepackage[left=1in,right=1in,top=1in,bottom=1in]{geometry}
\usepackage{amsmath}
\usepackage{amsthm} %proof environment
\usepackage{amssymb}
\usepackage{amsfonts}
\usepackage{enumitem} %nice lists
\usepackage{verbatim} %useful for something 
\usepackage{xcolor}
\usepackage{setspace}
\usepackage{titlesec}
\usepackage{blindtext} % I have no idea what this is 
\usepackage{caption}  % need this for unnumbered captions/figures
\usepackage{natbib}
\usepackage{tikz}
\usepackage{hyperref}


\titleformat{\section}{\bfseries\Large}{Problem \thesection:}{5pt}{}
\begin{document}

\title{AM 275 - Magnetohydrodynamics: Homework 1}
\author{Dante Buhl}


\newcommand{\wrms}{w_{\text{rms}}}
\newcommand{\bs}[1]{\boldsymbol{#1}}
\newcommand{\tb}[1]{\textbf{#1}}
\newcommand{\bmp}[1]{\begin{minipage}{#1\textwidth}}
\newcommand{\emp}{\end{minipage}}
\newcommand{\R}{\mathbb{R}}
\newcommand{\C}{\mathbb{C}}
\newcommand{\N}{\mathcal{N}}
\newcommand{\K}{\bs{\mathrm{K}}}
\newcommand{\m}{\bs{\mu}_*}
\newcommand{\s}{\bs{\Sigma}_*}
\newcommand{\dt}{\Delta t}
\newcommand{\dx}{\Delta x}
\newcommand{\tr}[1]{\text{Tr}(#1)}
\newcommand{\Tr}[1]{\text{Tr}(#1)}
\newcommand{\Div}{\nabla \cdot}
\renewcommand{\div}{\nabla \cdot}
\newcommand{\Curl}{\nabla \times}
\newcommand{\Grad}{\nabla}
\newcommand{\grad}{\nabla}
\newcommand{\grads}{\nabla_s}
\newcommand{\gradf}{\nabla_f}
\newcommand{\xs}{x_s}
\newcommand{\xf}{x_f}
\newcommand{\ts}{t_s}
\newcommand{\tf}{t_f}
\newcommand{\pt}{\partial t}
\newcommand{\pz}{\partial z}
\newcommand{\uvec}{\bs{u}}
\newcommand{\F}{\bs{F}}
\newcommand{\T}{\tilde{T}}
\newcommand{\ez}{\bs{e}_z}
\newcommand{\ex}{\bs{e}_x}
\newcommand{\ey}{\bs{e}_y}
\newcommand{\eo}{\bs{e}_{\bs{\Omega}}}
\newcommand{\ppt}[1]{\frac{\partial #1}{\partial t}}
\newcommand{\DDt}[1]{\frac{D #1}{D t}}
\newcommand{\ppts}[1]{\frac{\partial #1}{\partial t_s}}
\newcommand{\pptf}[1]{\frac{\partial #1}{\partial t_f}}
\newcommand{\ppz}[1]{\frac{\partial #1}{\partial z}}
\newcommand{\ddz}[1]{\frac{d #1}{d z}}
\newcommand{\ppzetas}[1]{\frac{\partial^2 #1}{\partial \zeta^2}}
\newcommand{\ppzs}[1]{\frac{\partial #1}{\partial z_s}}
\newcommand{\ppzf}[1]{\frac{\partial #1}{\partial z_f}}
\newcommand{\ppx}[1]{\frac{\partial #1}{\partial x}}
\newcommand{\ppxi}[1]{\frac{\partial #1}{\partial x_i}}
\newcommand{\ppxj}[1]{\frac{\partial #1}{\partial x_j}}
\newcommand{\ppy}[1]{\frac{\partial #1}{\partial y}}
\newcommand{\ppzeta}[1]{\frac{\partial #1}{\partial \zeta}}


\maketitle 
% This line removes the automatic indentation on new paragraphs
\setlength{\parindent}{0pt}

\section{}

Show that 
\begin{gather*}
    u_i\ppxj{\tau_{ij}} = \ppxj{u_i\tau_{ij}} + pe_{kk} -
    2\mu\left[e_{ij} - \frac{1}{3}e_{kk}\delta_{ij}\right]^2. 
\end{gather*}

\begin{proof}
    First, we begin with the derivative identity
    \begin{gather*}
        u_i\ppxj{\tau_{ij}} = \ppxj{u_i\tau_{ij}} - \tau_{ij}\ppxj{u_i}
    \end{gather*}
    and in order to simplify this statement, take $\tau_i$ to be the i-th row
    vector of $\tau$, we have:
    \begin{gather*}
        \sum_i u_i\div{\tau_i} = \sum_i\div{u_i\tau_i} - \tau_i\cdot\grad u_i
    \end{gather*}
    Already we have shown the first RHS term originates from the derivative
    identity, whereas the other terms must originate from
    $-\sum_i\tau_i\cdot\grad u_i$. Thus, we investigate this term in more
    detail. 
    \begin{gather*}
        -\sum_i\tau_i\cdot\grad u_i =\sum_i \left[p +
        \frac{2}{3}\mu\grad\cdot\uvec\right]\delta_{ij}\cdot\grad u_i - 2\mu
        e_i\cdot\grad u_i
    \end{gather*}
    where $e_{kk}$ is written as $\grad\cdot\uvec$ and $e_i$ is the i-th row of
    $e$ (as in $e_{ij}$). Notice that $\sum_{i}\delta_{ij}\cdot\grad u_i =
    \grad\cdot\uvec$, and therefore, 
    \begin{align*}
         -\sum_i\tau_i\cdot\grad u_i &= \left[p +
        \frac{2}{3}\mu\grad\cdot\uvec\right](\grad\cdot\uvec) - 2\mu
        \sum_ie_i\cdot\grad u_i\\
        &= p(\div{\uvec}) + \frac{2}{3}\mu(\div{\uvec})^2 - 2\mu
        \sum_ie_i\cdot\grad u_i
    \end{align*}
    Thus we recover the second RHS term, $pe_{kk}$. Now we must show the rest of
    $-\sum_i\tau_i\cdot\grad u_i$ recovers the last term of the RHS. We write
    the decomposition of $e_{i}$. 
    \begin{align*}
        -2\mu\sum_ie_i\cdot\grad u_i &= -\mu\sum_i \left(\grad u_i +
        \ppxi{\uvec}\right)\cdot\grad u_i\\
        &= -\mu \sum_i |\grad u_i|^2 + \ppxi{\uvec}\cdot\grad u_i\\
        &= -\mu |\grad\uvec|^2 - \mu\sum_i \ppxi{\uvec}\cdot\grad u_i\\
        &= -\mu |\grad\uvec|^2 - \mu\left((\grad\cdot\uvec)^2 + 2\left(\ppx{v}\ppy{u}
        + \ppx{w}\ppz{u} + \ppy{w}\ppz{v}\right)\right)
    \end{align*}
    Now we must show by the transitive propery that,
    \begin{align*}
        \frac{2}{3}\mu(\div{\uvec})^2 - \mu |\grad\uvec|^2 - \mu\left((\grad\cdot\uvec)^2 + 2\left(\ppx{v}\ppy{u}
        + \ppx{w}\ppz{u} + \ppy{w}\ppz{v}\right)\right) &= -
        2\mu\left[e_{ij} - \frac{1}{3}e_{kk}\delta_{ij}\right]^2\\
        -\frac{1}{3}(\div{\uvec})^2 - |\grad\uvec|^2 - 2\left(\ppx{v}\ppy{u}
        + \ppx{w}\ppz{u} + \ppy{w}\ppz{v}\right) &= -
        2\left[e_{ij} - \frac{1}{3}e_{kk}\delta_{ij}\right]^2
    \end{align*}

    \begin{gather*}
        -2\left[e_{ij} - \frac{1}{3}e_{kk}\delta_{ij}\right]^2=-2\left[e_{ij}^2 - \frac{2}{3}(\div{\uvec})e_{ij} -
        \frac{1}{9}(\div{\uvec})^2\bs{I}\right]
    \end{gather*}
\end{proof}






\end{document}
