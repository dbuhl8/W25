\documentclass{article}

\usepackage{graphicx} % Required for inserting images
\usepackage[left=1in,right=1in,top=1in,bottom=1in]{geometry}
\usepackage{amsmath}
\usepackage{amsthm} %proof environment
\usepackage{amssymb}
\usepackage{amsfonts}
\usepackage{enumitem} %nice lists
\usepackage{verbatim} %useful for something 
\usepackage{xcolor}
\usepackage{setspace}
\usepackage{titlesec}
\usepackage{blindtext} % I have no idea what this is 
\usepackage{caption}  % need this for unnumbered captions/figures
\usepackage{natbib}
\usepackage{tikz}
\usepackage{hyperref}

\titleformat{\section}
{\bfseries\Large}
{Lecture \thesection:}
{5pt}{}

\begin{document}

\title{AM 275 - Magnetohydrodynamics: Lecture Notes}
\author{Dante Buhl}
\doublespacing



\newcommand{\wrms}{w_{\text{rms}}}
\newcommand{\bs}[1]{\boldsymbol{#1}}
\newcommand{\tb}[1]{\textbf{#1}}
\newcommand{\bmp}[1]{\begin{minipage}{#1\textwidth}}
\newcommand{\emp}{\end{minipage}}
\newcommand{\R}{\mathbb{R}}
\newcommand{\C}{\mathbb{C}}
\newcommand{\N}{\mathcal{N}}
\newcommand{\K}{\bs{\mathrm{K}}}
\newcommand{\m}{\bs{\mu}_*}
\newcommand{\s}{\bs{\Sigma}_*}
\newcommand{\dt}{\Delta t}
\newcommand{\dx}{\Delta x}
\newcommand{\tr}[1]{\text{Tr}(#1)}
\newcommand{\Tr}[1]{\text{Tr}(#1)}
\newcommand{\Div}{\nabla \cdot}
\renewcommand{\div}{\nabla \cdot}
\newcommand{\Curl}{\nabla \times}
\newcommand{\Grad}{\nabla}
\newcommand{\grad}{\nabla}
\newcommand{\grads}{\nabla_s}
\newcommand{\gradf}{\nabla_f}
\newcommand{\xs}{x_s}
\newcommand{\xf}{x_f}
\newcommand{\x}{\bs{x}}
\newcommand{\ts}{t_s}
\newcommand{\tf}{t_f}
\newcommand{\pt}{\partial t}
\newcommand{\pz}{\partial z}
\newcommand{\uvec}{\bs{u}}
\newcommand{\jvec}{\bs{j}}
\newcommand{\bvec}{\bs{B}}
\newcommand{\B}{\bs{B}}
\newcommand{\evec}{\bs{E}}
\newcommand{\E}{\bs{E}}
\newcommand{\vort}{\bs{\omega}}
\newcommand{\F}{\bs{F}}
\newcommand{\T}{\tilde{T}}
\newcommand{\ez}{\bs{e}_z}
\newcommand{\ex}{\bs{e}_x}
\newcommand{\ey}{\bs{e}_y}
\newcommand{\eo}{\bs{e}_{\bs{\Omega}}}
\newcommand{\ppt}[1]{\frac{\partial #1}{\partial t}}
\newcommand{\pp}[2]{\frac{\partial #1}{\partial #2}}
\newcommand{\ddt}[1]{\frac{d #1}{d t}}
\newcommand{\DDt}[1]{\frac{D #1}{D t}}
\newcommand{\DD}[2]{\frac{D #1}{D #2}}
\newcommand{\ppts}[1]{\frac{\partial #1}{\partial t_s}}
\newcommand{\pptf}[1]{\frac{\partial #1}{\partial t_f}}
\newcommand{\ppz}[1]{\frac{\partial #1}{\partial z}}
\newcommand{\ddz}[1]{\frac{d #1}{d z}}
\newcommand{\ppzetas}[1]{\frac{\partial^2 #1}{\partial \zeta^2}}
\newcommand{\ppzs}[1]{\frac{\partial #1}{\partial z_s}}
\newcommand{\ppzf}[1]{\frac{\partial #1}{\partial z_f}}
\newcommand{\ppx}[1]{\frac{\partial #1}{\partial x}}
\newcommand{\ppxi}[1]{\frac{\partial #1}{\partial x_i}}
\newcommand{\ppxj}[1]{\frac{\partial #1}{\partial x_j}}
\newcommand{\ppy}[1]{\frac{\partial #1}{\partial y}}
\newcommand{\ppzeta}[1]{\frac{\partial #1}{\partial \zeta}}


\maketitle 
\pagebreak

\tableofcontents
\pagebreak

\setlength{\parindent}{0pt}
\setcounter{section}{1}

\section{Hydrodynamics Review}

What is a fluid?
\begin{itemize}
    \item Its flows! 
    \item it deforms continuously
\end{itemize}

Categorization of fluids:
\begin{itemize}
    \item Compressible v. incompressible
    \item viscous v. inviscid
    \item + many more
\end{itemize}

\vspace{20pt}

\bmp{.49}
    \centering
    {\Large \textbf{Eularian}}
    \vspace{5pt}

    Rate of change at a given point, no bother for where the fluid goes. 
    \begin{gather*}
        \frac{\partial}{\partial t}\left(\cdot\right) 
    \end{gather*}
\emp
\hspace{5pt}
\bmp{.49}
    \centering
    {\Large \textbf{Lagrangian}}
    \vspace{5pt}

    Follows the particle, introduces the advection term
    \begin{gather*}
        \uvec\cdot\grad \left(\cdot\right)
    \end{gather*}
\emp

\vspace{20pt}
{\Large \textbf{Mass Conservation}}
\vspace{5pt}


In order to conserve mass we consider an arbitrary eularian volume (i.e. the
volume doesn't move with the flow). We then find the total mass which is equal
to the integral of the density over the volume, and then consider the flux of
mass through the boundary (change in mass over time). Using the divergence
theorem, we then have a conservation equation for mass. 

\begin{gather*}
    \ppt{}\int_D\rho dV = \int_{\partial D} \rho\uvec \cdot \eta dA\\
    \ppt{\rho} + \grad\cdot \rho\uvec = 0 \\
    \ppt{\rho} + \uvec\cdot\grad\rho + \rho\grad\cdot\uvec = 0\\
    \frac{D\rho}{Dt} = - \rho \grad\cdot\uvec
\end{gather*}

More importantly, if we consider an incompressible fluid, i.e. $\rho = \rho_0$,
we have very specifically,
\begin{gather}
    \grad \cdot \uvec = 0
\end{gather}


\vspace{20pt}
{\Large \textbf{Stresses}}
\vspace{5pt}

Stresses can be divided into two catagories, body forces and surface forces.
Body forces are forces such as gravity and the electric force, which surface forces are
forces such as normal force and friction. 

\vspace{20pt}
{\Large \textbf{Newtons Second Law}}
\vspace{5pt}
Newton's second law 
\begin{gather*}
    \ppt{p} = \sum_i F_i
\end{gather*}
where $p$ here is the momentum of a fluid parcel. In actuality, the momentum can
be written as $p = \int_D \rho\uvec dV$. So, 
\begin{gather*}
    \DDt{}\int_D \rho\uvec dV = \int_D \rho \bs{g} dV + (\text{other body
    forces}) + \grad \cdot \bs{\tau}\\
    \rho\DDt{\uvec} = \rho F + \grad\cdot\bs{\tau}
\end{gather*}
where $\bs{\tau}$ is the stress tensor acting on the fluid parcel. Surface
forces are then introduced into this stress tensor. First and foremost, surface
pressure is introduced along the stress tensor. 
\begin{gather*}
    \tau_{ij} = -p\delta_{ij} + \sigma_{ij}
\end{gather*}
where $\sigma_{ij}$ is the deviatoric stress tensor and is responsible for the
off-diagonal components of the stress tensor. Some components are the velocity
gradient tensors, $\ppxj{u_i}$ and $\ppxi{u_j}$. Each of these has a symmetric
compoenent and an antisymmetric componenet. 
\begin{gather*}
    \ppxj{u_i} = \frac{1}{2}\left(\ppxj{u_i} + \ppxi{u_j}\right) +
    \frac{1}{2}\left(\ppxj{u_i} - \ppxi{u_j}\right)
\end{gather*}
The first term is labeled symmetric and denoted $e_{ij}$ while the second
component is the rotation componenet. 



\section{Continuing Review of the Kinematic Equations}

\subsection{Obtaining the Navier-Stokes equation}

Decomposition of the deviatoric stress tensor reveals a 4th order tensor with 81
components.
\begin{gather*}
    \bs{\sigma}_{ij} = \bs{A}_{ijkl}\bs{e}_{kl}
\end{gather*}
In order to reduce the complexity of the system, we make some assumptions about
the tensor $\bs{A}_{ijkl}$. First, we state that this tensor must be isotropic,
i.e. that it doesn't care about the direction of the stress with respect to the
coordinate system it is in. We have, 
\begin{gather*}
    \bs{A}_{ijkl} = \mu \bs{\delta}_{ij}\bs{\delta}_{kl} +
    \mu'\bs{\delta}_{ik}\bs{\delta}_{jl} +
    \mu''\bs{\delta}_{il}\bs{\delta}_{jk}
\end{gather*}
Next, we assume that this tensor must be symmetric. This reduces the complexity
down to two coefficients, $\mu$, the viscosity, and $\mu'$ which is the bulk viscosity. 

In order to obtain the Navier-Stokes equation, we require the Stokes assumption
which postulates that the diagonal components of the deviatoric stress tensor
are zero, i.e. $\bs{\sigma}_{ii} = 0$. 
\begin{gather*}
    \bs{\sigma}_{ij} = 2\mu\left(\bs{e}_{ij} -
    \frac{1}{3}\left(\grad\cdot\uvec\right)\bs{\delta}_{ij}\right)    \\
    \bs{\tau}_{ij} = -p\bs{\delta}_{ij} + 2\mu\left(\bs{e}_{ij} -
    \frac{1}{3}\left(\grad \cdot \uvec\right)\bs{\delta}_{ij}\right)
\end{gather*}
Therefore, when we take the divergence of this stress tensor we obtain the
Navier-Stokes equation:
\begin{gather*}
    \rho\DDt{\uvec} = \rho\bs{F} - \grad p + \mu\left[ \grad^2 \uvec +
    \frac{1}{3}\grad \left(\grad \cdot \uvec\right)\right]
\end{gather*}
Of course, when working in an incompressible framework (i.e. $\grad\cdot\uvec =
0$), we have that part of the diffusive term disappears from the equation,
resulting in the commonly used equation:
\begin{gather}
    \label{eq:NS}
    \DDt{\uvec} = \bs{F} - \frac{1}{\rho_0}\grad p + \nu\grad^2 \uvec
\end{gather}
Additional terms are included in this equation as necessary to model relevant
physics of various fluid systems. For example, if in a rotating frame we include
the coriolis force $2\Omega(\bs{e}_{\Omega} \times \uvec)$, if some component of
the fluid is stratified we need some buoyancy forcing $T/N^2\bs{e}_z$. And most
relevant, there might be magnetic forces which affect the fluid, in which case
we obtain the MHD equations. 

\subsection{Vorticity equation}

Vorticity is a quantity related to the fluid field which can be very important
to the scientific study of fluid dynamics. The vorticity is obtained by taking
the curl of the velocity field. 
\begin{gather*}
    \bs{\omega} = \grad \times \uvec
\end{gather*}
The vorticity has an evolution-advection equation just as the velocity field
does, and in fact the vorticity equation is obtained by taking the curl of the
Navier-Stokes equations.

\begin{gather*}
    \grad\times\eqref{eq:NS}\\
    \DDt{\vort} =  \left(\vort\cdot\grad\right)\uvec +
    \vort\left(\grad\cdot\uvec\right) + \grad\times\F -
    \frac{1}{\rho^2}\grad\rho\times\grad p + 
    \nu\grad^2\vort 
\end{gather*}

If the flow is incompressible, one of the vortex stretching terms disappears.
Generally, the first two right hand terms are vortex stretching/tilting/speed-up
terms. Then the pressure and density gradient cross product is the baroclinicity
term, the curl of $\F$ is the forcing of vorticity, and finally, we have a
viscous diffusion of vorticity which behaves similarly to the diffusion of
velocity. 


Baroclinicity is perhaps the most unintuitive term in this equation, and it
simply represents the creation of rotation in the fluid when there is a
disalignment between the pressure and density gradients in the fluid. 

Some fluid dynamicists prefer to study the vorticity equation, especially for
rotating flows where voricites and cyclones are common phenomenon in the flow
field. 

\subsection{Rotation}

In the presence of rotation, the coriolis force becomes relevant as the motion
of a fluid particle is deflected due to the rotation of the cordinate frame.
That is, our equations are modified such that, 
\begin{gather*}
    \ppt{\bs{q}}_F = \ppt{\bs{q}}_R + 2\bs{\Omega}\times \bs{q} -
    \bs{\Omega}^2\bs{R}
\end{gather*}

This also introduces an additional term to the vorticity equation which looks
like, $+\left(2\bs{\Omega}\cdot\grad\right)\uvec$. 


\section{Conservation of Energy and Maxwell's equations}

\subsection{Conservation of Energy}

The equation of state chosen for a particular problem is a source of physics
which affects the solutions of a given PDE. The incompressible equation of
state is used very commonly as an equation of state. Another common one is the
ideal gas law $pV = \rho RT$. 

In order to understand the origin and importance of the equation of state, the
laws of thermodynamics are needed. 

The first law of thermodynamics states, 

\begin{gather*}
    \ppt{e} = \ppt{W} + \ppt{Q}
\end{gather*}
where $e$ is the internal energy, $W$ is work done on the system, and $Q$ is
heat flux into the system. However, for a fluid flow taken from a Lagrangian
perspective, we must modify this law of thermodynamics. It must include the
energy given by the velocity field. 

\begin{gather*}
    \DDt{}\int_D \rho\left(e+\frac{1}{2}\uvec^2\right)dV = \int_D \rho
    \F\cdot\uvec dV + \int_{\partial D} \bs{\tau}\cdot\uvec dS - \int_{\partial
    D} \bs{q}\cdot dS \\
    \rho\DDt{}\left(e+\frac{1}{2}\uvec^2\right) = \rho\F\cdot\uvec +
    \grad(\bs{\tau}\cdot\uvec) -
    \grad\cdot\bs{q} 
\end{gather*}

Next, we obtain a mechanical energy equation by dotting $\uvec$ by the
Navier-Stokes equation and adding $\uvec^2/2 \cdot \DDt{\rho}$
\begin{gather*}
    \DDt{\rho\uvec^2/2} = \rho\F\cdot\uvec -
    \uvec\cdot\left(\grad\cdot\bs{\tau}\right) + \ldots\\
    \uvec\cdot\left(\grad\cdot\bs{\tau}\right) = 
    \Phi = 2\mu\left[\bs{e}_{ij} -
    \frac{1}{3}(\grad\cdot\uvec)\bs{\delta}_{ij}\right]
\end{gather*}

Finally, we obtain an energy equation with a positive definite dissipation term
$\Phi$ which acts purely to remove energy from the system. 
\begin{gather*}
    \rho\DDt{e} = -\grad \cdot \bs{q} - p\left(\grad\cdot\uvec\right) + \Phi
\end{gather*}

The Second law of Thermodynamics also plays an important role in the
conservation of energy. The ssecond law makes statements about the entropy of a
system, $S$.
\begin{gather*}
    dS = \frac{dq}{T}\\
    TdS =  de + pdV\\
    T\ddt{S} = \ddt{e} - \frac{p}{\rho^2}\ddt{\rho}\\
    \rho \DDt{S} = - \frac{\grad \cdot \bs{q}}{T} + \frac{k}{T^2}|
    \grad T|^2 + \mu\frac{\Phi}{T}
\end{gather*}
Essentially, since both $|\grad T|^2$ and $\Phi$ are positive definite terms and
their coefficients are positive definite, it must be that the entropy of a
system can only increase ``on average'' (curse the statisticians). 

\subsection{Introducing Maxwell's Equations}

Electricity and Magnetism are very closely related to one another and governed
by a main set of governing equations. The main variables which we consider are a
position vector, $\bs{x}$, a velocity field, $\uvec$, density $\rho$, pressure
$p$, time $t$, temperature $T$, magnetic field (magnetic flux density) $\bvec$,
magnetic field strength $\bs{H}$, electric field $\bs{E}$, electric displacement
$\bs{D}$, electric current density $\bs{j}$, and charge density $\rho_e$. 

Alongside these variables, we have constants describing components of
electropmangetism: permitivity $\varepsilon$, permeability $\mu$, and
conductivity $\sigma$. Permitivity describes the charge requirement for a
specific electric field, i.e. large $\varepsilon$ implies a larger charge is
needed for a specific electric field. Permeability describes the current
requirement for a specific magnetic field, i.e. large $\mu$ implies a smaller
current is needed to obtain a specific magnetic field. 

Consitutive relationships describe the relationships between specific
electromagnetic quantities. 
\begin{gather*}
    \bs{H} = \frac{\bvec}{\mu}, \text{ for an isotropic permeability}\\
    \bs{D} = \varepsilon \bs{E}, \text{ for an isotropic permitivity}
\end{gather*}
where generally, we take $\mu = \mu_0$ and $\varepsilon = \varepsilon_0$ where
$q_0$ is taken from a vacuum. 

Now we write Maxwell's equations in their differential form:
\begin{align*}
    &\grad\cdot\bvec = 0, \text{ Gauss' law for magnetism}
    & \grad \cdot \bs{E} = \frac{\rho_e}{\varepsilon_0}, \text{ Gauss' law}\\
    &\grad \times \bs{E} = -\ppt{\bvec}, \text{ Faraday's law}
    &\grad \times \bvec = \mu_0\left(\bs{j} + \varepsilon_0\ppt{\bs{E}}\right),
    \text{ Ampere's Law of induction}
\end{align*}

They can be written in their integral form as well: 
\begin{align*}
    &\oint_{\partial D} \bvec\cdot\bs{dA} = 0 
    & \oint_{\partial D} \bs{E}\cdot\bs{dA} = \frac{q}{\varepsilon_0}\\
    &\oint_L \bs{E}\cdot\bs{dL} = -\ppt{\phi_{B}} 
    & \oint_L\bvec\cdot\bs{dL} = \mu_0\bs{j} + \mu_0\varepsilon_0\ppt{\phi_E}
\end{align*}
where $\phi_B = \int_S \bvec\cdot\bs{dA}$ is the total magnetic fluid, and 
$\phi_E = \int_S \bs{E}\cdot\bs{dA}$ is the total electric flux. 


We cover their derivations in a brief sense also. Consider a positive point
charge which creates an electric field. This imposes a force acting on any other
point charge in the field. This force is called Coulomb force given by $F =
q_1q_2/(4\pi\varepsilon r^2)$. Thus we have a given electric field of strength
$E/q = q_1/(4\pi\varepsilon r^2)$. We obtain the total electric flux $\phi_E$ 

\begin{gather*}
    d\phi_E = \bs{E}\cdot\bs{dA}\\
    \phi_E = \oint_S\frac{q_1}{4\pi\varepsilon r^2}\cdot\bs{dA}\\
    \phi_E = \frac{q_1}{4\pi\varepsilon r^2}\oint_S\bs{dA}\\
    \phi_E = \frac{q_e}{\varepsilon}
\end{gather*} 
where $q_e$ in the final equation is given by the sum of all point charges
enclosed in the closed volume, i.e. $q_e = \sum_i q_i$. Notice that $q_e$ can be
thought of as mass for point charges, i.e. the integral of the charge density
equals the total charge similar to how the integral of mass density equals the
total mass. It can be represented as a sum of point charges because point
cahrges are discrete and do not usually exist in a continuum. 

Finally, using the divergence theorem, 
\begin{gather*}
    \oint_S \bs{E}\cdot\bs{dA} = \int_V \grad\cdot\bs{E} dV\\
    q_e = \int_V \rho_{E} dV\\
    \grad\cdot\bs{E} = \frac{\rho_E}{\varepsilon}
\end{gather*}
Similarly, the same proof holds for Gauss' law of magnetism, only that monopoles
do not exist in magnetic fields, i.e. every source must have a sink. Therefore,
for an arbitraty volume it must be that the divergence of the magnetic field
must be zero: 
\begin{gather*}
    \div\bvec = 0
\end{gather*}

\section{Derivation of Maxwell's Equations: Continued}

\subsection{Faraday's Law of Induction}
    
The laws of electrodynamics are empiracle. Faraday realized that the EMF,
proportional to $\ppt{\bvec}$ and also the area of the coil. This led them to
deduce that EMF should be proportional to $\ppt{\phi_B}$. Note that EMF
represents the amount of work done per unit charge to move a charge from one
place to another, i.e. the electric potential difference. It has the units of
$Nm/C$ (Newton meters per Coulomb). 
\begin{align*}
    \text{EMF} = \oint_C\bs{E}\cdot\bs{dL} &= - \ppt{\phi_B} \\
    &= -\ppt{}\int_S\bvec\cdot\bs{dA}
\end{align*}
where here the RHS integral is not necessarily over a closed surface (because
the same integral over a closed surface must be zero). 

\subsection{Ampere's Law}

If there is a current moving through a wire, imagine a cross section going
through the wire (i.e. going through the page), there is a magnetic field around
the wire. The magnetic field can be described using the following integral form. 
\begin{gather*}
    \oint_L \bs{H}\cdot\bs{dL} = \bs{i}\\
    \oint_L\frac{\bvec}{\mu_0}\cdot\bs{dL} = \bs{i}\\
    \oint_L\bvec\cdot\bs{dL} = \mu_0\bs{i}\\
    \int_S\grad\times\bvec\cdot\bs{dA} = \mu_0\oint_A\bs{j}\cdot\bs{dA}\\
    \grad\times\bvec = \mu_0\bs{j}
\end{gather*}
Here is where Maxwell's contribution to Ampere's law is notable. Ampere assumed
that $\div{\bs{j}} = 0$, where Maxwell noticed that in some scenarios, this
is not necessarily true. Thus, he modified the equation to include displacement
currents. 

\begin{gather*}
    \ppt{\bs{D}} = \bs{j}_0\\
    \ppt{\varepsilon \bs{E}} = \bs{j}_0\\
    \grad\times\bvec = \bs{j} + \bs{j}_0\\
    \grad\times\bvec = \mu_0\bs{j} + \mu_0\varepsilon_0\ppt{\bs{E}}
\end{gather*}
This was significant because $\mu_0\varepsilon_0 \propto 1/c^2$ (where $c$ is
the speed of light), and this
implied connections to electromagnetic radiation (doublecheck this). More
importantly, these equations are linear and relativistically correct (not sure
what exactly this means). 

\subsection{Units of Electrodynamics (c.f. Priest p436)}

Electrostatis units are denoted ``esu''. Electromagnetic units are denoted
``emu'': e, m respectively. The Gaussian cgs system utilizes the standard units
of centimeters, grams, seconds, in addition to the electrostatic units
statcoulomb, q, and electromagnetic units, "abAmp". The Gaussian cgs
representation of the governing equations often have an extra factor of $4\pi$
in the equations. 

In general, we will use the Rationalized MKS system (standard SI system). Where
the default length, mass, time, is given in meters, kilograms, and seconds. In
addition, current is given in amps. The variables, $\mu_0 = 4\pi\cdot10^{-7}
NA^{-2}$
and $\varepsilon_0 = 8.8\cdot10^{-12} A^2s^2N^{-1}m^{-1}$ have dimension,
charges are given in Coulombs, forces are given in Newtons. The magnetic field
is given by Teslas $T = NA^{-1}m^{-1}$, and the electric field is given by
$V/m$ (Volts per meter). 

\subsection{From Maxwell's Equation to MHD}

Generally, for MHD we will be working in a non-relativistic approximation (i.e.
typical velocities are much less than the speed of light, $U \ll c$). Let us
consider the equations and their typical unit scales,

\begin{gather*}
    \grad\times\bs{E} = -\ppt{\bvec}\\
    \frac{\E}{L} ~ \frac{\B}{T}\\
    \grad\times\B = \mu_0\bs{j} + \frac{1}{c^2} \ppt{\E}\\
    \frac{\B}{L} ~ \frac{1}{c^2}\frac{\B L}{T^2}
\end{gather*}
If we manipulate the last line of this equation, we find that $L^2/T^2 ?= c^2$
is the leading balance of Ampere's law, and therefore we neglect the
relativistic term of Maxwell's equations. 

In order to connect Maxwell's equations to fluid dynamics, we must consider the
Lorentz force. 
\begin{gather*}
    \F = q\left(\E + \uvec\times\B\right)\\
    \frac{d\F}{d V} = \frac{dq}{dV}\left(\E + \uvec\times\B\right)\\
    \frac{d\F}{d V} = \rho_E\left(\E + \uvec\times\B\right)\\
    \frac{d\F}{d V} = \rho_E\E + \bs{j}\times\B\\
    \F = \int_V \rho_E\E + \bs{j}\times\B dV
\end{gather*}
Next we must consider Ohm's Law, which describes the current and electric field
as movign with the conductor (Lagrangian perspective) (denoted with $'$). 
\begin{gather*}
    \bs{j}' = \sigma\E'
\end{gather*}
and thus we are able to simplify the equations to become, 
\begin{gather*}
    \E' = \E + \uvec\times\B\\
    \bs{j}' = \jvec\\
    \jvec = \sigma(\E + \uvec\times\B)\\
    \grad\times\B = \mu_0\jvec\\
    \grad\times\B = \mu_0\sigma(\E + \uvec\times\B)
\end{gather*}
Taking the curl of this equation leads to the following,
\begin{gather*}
    \grad\times\left(\frac{\grad\times\B}{\mu_0\sigma}\right) = \grad\times\E +
    \grad\times\left(\uvec\times\B\right)\\
    \grad\times\eta\grad\times\B = -\ppt{\B} + \grad\times(\uvec\times\B)\\
    \ppt{\B} = -\grad\times(\eta\grad\times\B) + \grad\times(\uvec\times\B)
\end{gather*}
which is the induction equation. If we take $\eta$ to be constant, we can write
with the derivative identity, $\grad\times\grad\times\B = \grad(\div{\B}) -
\grad^2\B$:
\begin{gather*}
    \ppt{\B} = \grad\times(\uvec\times\B) + \eta\grad^2\B
\end{gather*}
Notice that we obtain an equation solely for $\B$ which has taken into account
all of Maxwell's equations. This tells us that we really only have to care about
the magnetic field, and can obtain the electric field as a consequence of our
solution. For example, $\jvec = \grad\times\B/\mu_0$, $\E=\jvec/\sigma -
\uvec\times\B$, and $\rho_E = \varepsilon_0(\div{\E})$. 

We can interpret the terms in this equation as well. On the LHS we have a
typical rate of change of the magnetic field. On the RHS we have first the
induction term, and the diffusion of the magnetic field $\B$. 

We also notice that the linearity of this equation depends primarily on the
relationship between $\uvec$ and $\B$. If, for example, $\uvec$ is a function of
$\B$ then the induction equation is not linear. If the induction equation is
linear, then the equation is generally regarded as the kinematic induction
equation. If the equation is nonlinear, then it is generally regarded as a
dynamic equation of induction. 

In general, the Lorentz force is vital to determining which dynamical regime we
are in for the velocity and magnetic fields. Consider again the Lorentz force, 
\begin{gather*}
    \F = \rho_E \E + \jvec\times\B
\end{gather*}
where the first RHS term is the electrostatic component and the second RHS term
is the magnetic component. Generally, we compare the order of each term in the
equation. 
\begin{gather*}
    \frac{|\rho_E\E|}{|\jvec\times\B|} \propto
    \frac{|\varepsilon_0\div{\E}\E|}{|(\grad\times\B)\B/\mu_0|}\\
    \propto \frac{\varepsilon_0\mu_0\E^2/L}{\B^2/L} \\
    \propto \varepsilon_0\mu_0\left(\frac{L}{T}\right)^2 = \frac{U^2}{c^2}
\end{gather*}
Therefore we are able to deduce that the Lorentz force in a non-relativistic
regime, can be approximated as:
\begin{gather*}
    \F \propto \jvec\times\B\\
    \propto \frac{1}{\mu_0}(\grad\times\B)\B
\end{gather*}

With this Lorentz force as a body force, we write the Navier Stokes Equation
\begin{align*}
    \rho\DDt{\uvec} &= -\grad p + \mu\left[\grad^2\uvec +
    \frac{1}{3}\grad(\div{\uvec})\right] + \rho\F +
    \rho\left(\jvec\times\B\right)\\
    \DDt{\rho} + \rho(\grad\cdot\uvec) &= 0\\
    \DDt{e} &= \ldots\\
    &\text{An equation of state}\\
    \ppt{\B} &= \grad\times(\uvec\times\B) + \eta\grad^2\B
\end{align*}


\section{Boundary Conditions and Kinematics for MHD}

\subsection{Validity of MHD equations and assumptions}

\begin{enumerate}
\item
In MHD, we are attempting to treat a plasma as a fluid (a continuum), which is
not necessarily always valid. Length scales of interest for MHD problems are
much larger than lengthscales for plasma physics, i.e. ion gyroradius. If we
look at a small scale problem, this assumption becomes much less valid. (NOTE:
how does this resolve high Reynolds number flows, where the kolmogorov length
scale approaches typical plasma physics length scales? Ask Nic next class). 

\item The next consideration is how we represent plasma in the thermodynamic
equilibrium. For example, we require typical timescales and lengthscales to be
much larger than particle collision times and mean free path lengths. 

\item The constants $\eta$, $\mu$, $k$ are uniform, and isotropic, which is an
assumption we will take for granted, but there exist fluids who don't satisfy
these properties. 

\item The equations are in an inertial frame

\item Non-relativistic flows (because we disregarded Maxwell's addition to Ampere's
law). This requires that the flow speed is much less than the speed of light,
$c$. 

\item This derivation relies on a very simple version of Ohm's law, more
complicated forms of this physical law will not necessarily recover the
induction equation we have derived earlier. 

\item Plasma is a single fluid, i.e. plasmas can have Consitutive parts which
contribute to its total mass, and all of them don't necessarily behave the same.
Ideally, we would have some statisitical framework for the composition and
behavior of a plasma and incorporate that into our model. 


\end{enumerate}

To summarize what we have obtained so far, let us write the incompressible MHD
equations,
\begin{gather}
    \div{\uvec} = 0\\
    \rho\DDt{\uvec} = -\grad p + \F + \jvec\times\B + \mu\grad^2\uvec\\
    \DDt{\B} = (\B\cdot\grad)\uvec + \eta\grad^2\B
\end{gather}


\subsection{Consequences of $\div{\B} = 0$}

We have (and this can be shown easily) that if the property that the magnetic
field is divergence free as an initial condition, we have that this property is
maintained for all $t>0$. As a consequence, we can write that $\B$ as a
potential function, i.e. $\B = \grad\times\bs{A}$, where $\bs{A}$ is a vector
potential, 
\begin{gather*}
    \bs{A} = \grad\phi + A
\end{gather*}

An example of finding vector potentials in a spherical coordinate frame, is the
decomposition of a poloidal and toroidal magnetic field:
\begin{gather*}
    \B = \B_P + \B_T =\grad\times\left(\grad\times(P\bs{r}) +
    \grad\times(T\bs{r})\right)
\end{gather*}
where $P$ and $T$ are scalar functions which represent the poloidal and toroidal
components of the magnetic field respectively. In a physical sense, we can think
of the toroidal field being the axis-symmetric, azmuthal component, and the
poloidal field being the meridional component. 

\subsection{Common Boundary Conditions for the MHD equations}

Consider an interface between two mediums $M_1$ and $M_2$. Let us denote the
normal vector to that interface $\hat{\bs{n}}$. We can imagine a cylindrical
volume through the interface, which we will refer to as the ``pill-box''
(otherwise known as a Gaussian Box), and
this is often used for considering fluxes through the boundary. We can imagine
a contour line along the interface which is very thin and envelops a section of
the interface, particularly with a direction along the contour. This could be a
line integral or a surface integral condition for example. This is called "along
the contour" (also known as an Amperion Loop). 

The integral form of these boundary conditions can be written as,

\begin{gather*}
    \int_V \div{\B} dV = 0\\
    \int_S \B\cdot\bs{dS} = 0
\end{gather*}
AS this applies to the gaussian pill-box scenario, which is composed of three
surfaces, top and bottom $S_1$ and $S_3$, and the side $S_2$, eich of which has
their own unit normal vectors. 
We consider, 
\begin{gather*}
    \int_S \B\cdot\bs{dS} = \int_{S_1}\B\cdot\hat{\bs{n}}_1dS_1 +
    \int_{S_1}\B\cdot\hat{\bs{n}}_2dS_2 + \int_{S_3}\B\cdot\hat{\bs{n}}_3dS_3 =
    0
\end{gather*}
Next we consider the limit, where the height of the cylinder tends to zero (note
that the cylinder is centered along the interface). Therefore, the cylinder is
compressed to a circle on the surface of the interface, and specifically the
integral becomes,
\begin{gather*}
    \int_S \B\cdot\bs{dS} = 
    \int_{S_1}\B\cdot\hat{\bs{n}}_2dS_2  = 0
\end{gather*}
In essence, this allows us to formulate a ``jump condution'' that there must be
no discontinuity in the normal components of $\B$ or $\jvec$ through the surface of the interface. These
boundary conditions are often expressed as, 
\begin{gather*}
    \left[\B\cdot\hat{\bs{n}}\right] = 0,
    \quad\left[\jvec\cdot\hat{\bs{n}}\right] = 0
\end{gather*}

Next we must consider the tangential components, and this requires that we
revisit Faraday's law, 
\begin{gather*}
    -\ppt{}\int_S\B\cdot\bs{dS} = \oint_L \E\cdot\bs{dL} 
\end{gather*}
If we consider the surface to be an Amperion loop and take the height of that
loop to tend to zero, we have that the LHS of the given equation is zero, i.e. 
\begin{gather*}
    \oint_L \E\cdot\bs{dL} = -\int_{\text{top}} \E_{T1}\cdot\bs{dL_1} - 
    \int_{\text{left}} \E_{N1}\cdot\bs{dL_2} +\int_{\text{bottom}} \E_{T2}\cdot\bs{dL_3}
    +\int_{\text{right}} \E_{N2}\cdot\bs{dL_4} = 0\\
    \E_{T1} = \E_{T2}\\
    \left[\E \times \hat{\bs{n}}\right] = 0
\end{gather*}
we can then rewrite this using Ohm's law $\jvec = \sigma(\E + \uvec\times\B)$,
\begin{gather*}
    \left[\left(\frac{\jvec}{\sigma} - \uvec\times\B\right) \times \hat{\bs{n}}\right]
    = 0\\
    \left[\frac{\jvec}{\sigma} \times \hat{\bs{n}}\right] = 0
\end{gather*}
where the last line is obtained assuming that $\uvec = 0$ or that $\uvec$ is
purely tangential to the boundary. This is indicating in essence, that the jump
condition for $\jvec/\sigma$ must be satisfied across the interface, i.e.
$\jvec_1/\sigma_1 = \jvec_2/\sigma_2$. 

Another source of boundary conditions can come from Ampere's Law. We have, 
\begin{gather*}
    \int_S \frac{\grad\times\B}{\mu_0} \cdot\bs{dS} = \int_S \jvec\cdot\bs{dS} =
    I
\end{gather*}
where $I$ is the total enclosed current. 
\begin{gather*}
    \oint_L \frac{\B}{\mu_0} \cdot\bs{dL} = I
\end{gather*}
Taking the ``along the contour'' integral approach, we have,
\begin{gather*}
    \left[\frac{\B}{\mu_0} \times \hat{\bs{n}}\right] = \jvec_S
\end{gather*}
where $\jvec_S$ is the surface current density ($\approx I/dL$). If there is no
surface current, the RHS of this jump condition goes to zero and we have
similarly, $\B_1/\mu_1 = \B_2/\mu_2$. 

\subsection{Kinematics}

In orde to have a kinematic understanding of the MHD equations, we must have an
induitive understanding of the effects that magnetic field and field lines have
on the flows they generate. 

One can imagine, magnetic field lines which exist around a live wire. These
field lines have the same property as streamlines in hydrodynamics, i.e. they
give contours of constant field strength along the field line. These field lines
must be parallel to one another. We can imagine a small change in distance along
a field line $ds$ (which is essentially an infinitesimal arc length). We can
say then, 
\begin{gather*}
   \frac{d\bs{x}}{ds} = \B(\x,t_0)
\end{gather*}
where $\B(\x,t_0)$ is a snapshot of the magnetic field at $t = t_0$. 
We define $\B$ then as follows, 
\begin{gather*}
    \B = \left< \frac{dx}{ds}, \frac{dy}{ds},\frac{dz}{ds}\right>
\end{gather*}
Consider for example, a magnetic field given by $\B = \left<y, x, 0\right>$. We
would have then, 
\begin{gather*}
    \frac{dx}{\B_x} = \frac{dy}{\B_y} = ds\\
    \frac{dx}{y} = \frac{dy}{x} = ds\\
    \int xdx = \int ydy \\
    \frac{x^2}{2} = \frac{y^2}{2} + C\\
    x^2 - y^2 = C_1
\end{gather*}
where $C_1$ is given by the initial conditions for $x$ and $y$. We can imagine
what the field lines might look for this magnetic field, and they happen to look
like a saddle node in the $x-y$ plane. 

\section{Kinematics for MHD (Continued)}

\subsection{Magnetic Fieldlines}

\begin{gather*}
    \pp{\bs{x}}{s}(s) = \B(\x,t_0), \quad \x(s=0,t_0) = \x_0(t_0)
\end{gather*}

\subsection{Magnetic Flux}

\begin{gather*}
    d\bs{S} = \hat{\bs{n}}dS
\end{gather*}
We can define the Magnetic flux as the integral of $\B$ through the
surface/interface. 
\begin{gather*}
    \int_S\B\hat{\bs{n}}dS
\end{gather*}

\subsection{Magnetic Flux Tubes}

We have streamtubes for velocity, vortex tubes for vorticity, and magnetic flux tubes
for magnetic fields. Consequently, we must have that the flux through a surface
must be constant when that surface is advected some $\delta t$ away from its
original position, i.e.
\begin{gather*}
    \div{\B} = 0\\
    \int_{S1} \B\cdot\hat{\bs{n}}_1dS + \int_{S2} \B\cdot\hat{\bs{n}}_2dS +
    \int_{S3} \B\cdot\hat{\bs{n}}_3dS = 0\\
     \int_{S1} \B\cdot\hat{\bs{n}}_1dS = - \int_{S2} \B\cdot\hat{\bs{n}}_2dS
\end{gather*}
where $S1$ and $S2$ are the two surfaces at the end of the tube, and $S3$ is the
surface connecting the perimeters of $S1$ and $S2$. We can think of the LHS of
this equation as the flux coming in from the left, and the RHS as the flux
leaving from the right (or vice versa as I have written the signs here). 

\subsection{Nondimensionalization}

We remember first how to nondimensionalize the kinematic equations, first with
the navier stokes equation. 
\begin{gather*}
    \rho_0\DDt{\uvec} = -\grad p + \jvec\times\B +  \mu\grad^2\uvec\\
    \rho_0\frac{U^2}{L}\DD{\uvec'}{t'} = -\frac{P}{L}\grad' p' +
    \frac{B^2}{\mu_0L}(\jvec'\times\B') + 
    \mu\frac{U}{L^2}\grad'^2\uvec'\\
    \DD{\uvec'}{t'} = -\frac{P}{\rho_0U^2}\grad' p' +
    \frac{B^2}{\mu_0\rho_0U^2}(\jvec'\times\B') + 
    \frac{\mu}{\rho_0LU}\grad'^2\uvec'
\end{gather*}
where here we chose $P$ such that the fraction $PL/\rho_0U^2$ is equal to 1, and
define the Reynolds number $Re = U/\nu$, where $\nu = \mu/\rho_0$ and the
Chandrasehkar number $Q = B^2/\mu_0\rho_0U^2$ which describes the signficance of
the Lorentz force onto the dynamics of the system. 
\begin{gather*}
    \DD{\uvec'}{t'} = - \grad' p' + \frac{1}{Re}\grad'^2\uvec'
\end{gather*}
Next we nondimensionalize the induction equation,
\begin{gather*}
    \DDt{\B} = (\B\cdot\grad)\uvec + \eta\grad^2\B\\
    \frac{BU}{L} \DD{\B'}{t'} = \frac{BU}{L}(\B'\cdot\grad')\uvec' + \frac{\eta
    B}{L^2}\grad'^2\B'\\
    \DD{\B'}{t'} = (\B'\cdot\grad')\uvec' + \frac{\eta}{LU}\grad'^2\B'
\end{gather*}
where we define the Magnetic Renyolds number, $Rm = UL/\eta$. Thus we obtain
the nondimensionalized induction equation, 
\begin{gather*}
    \DD{\B'}{t'} = (\B'\cdot\grad')\uvec' + \frac{1}{Rm}\grad'^2\B'
\end{gather*}
Similar to the navier stokes equation, the Magnetic Reynolds number can be taken
to have two extreme limits in which the dynamics of the Magnetic field is
affected drastically. Most importantly, we have the limit where $Rm \gg 1$ in
which magnetic diffusivity is negligable and the induction equation can be
written as, 
\begin{gather*}
    \DDt{\B} = (\B\cdot\grad)\uvec
\end{gather*}
and conversely the opposite limit ($Rm \ll 1$) in which magnetic diffusivity is
dominant, i.e.
\begin{gather*}
    \ppt{\B} = \frac{1}{Rm}\grad^2\B.
\end{gather*}
In the diffusive case, the induction equation reduces to a heat-like
differential equation in which we have periodic spatial modes which decay on a
time scale dependent on the wavenumber of the spatial modes and the diffusion
coefficient $1/Rm \sim L^2/\eta$. 


\begin{gather*}
\end{gather*}

\end{document}
